%% Generated by Sphinx.
\def\sphinxdocclass{report}
\documentclass[letterpaper,10pt,english]{sphinxmanual}
\ifdefined\pdfpxdimen
   \let\sphinxpxdimen\pdfpxdimen\else\newdimen\sphinxpxdimen
\fi \sphinxpxdimen=.75bp\relax

\PassOptionsToPackage{warn}{textcomp}
\usepackage[utf8]{inputenc}
\ifdefined\DeclareUnicodeCharacter
 \ifdefined\DeclareUnicodeCharacterAsOptional
  \DeclareUnicodeCharacter{"00A0}{\nobreakspace}
  \DeclareUnicodeCharacter{"2500}{\sphinxunichar{2500}}
  \DeclareUnicodeCharacter{"2502}{\sphinxunichar{2502}}
  \DeclareUnicodeCharacter{"2514}{\sphinxunichar{2514}}
  \DeclareUnicodeCharacter{"251C}{\sphinxunichar{251C}}
  \DeclareUnicodeCharacter{"2572}{\textbackslash}
 \else
  \DeclareUnicodeCharacter{00A0}{\nobreakspace}
  \DeclareUnicodeCharacter{2500}{\sphinxunichar{2500}}
  \DeclareUnicodeCharacter{2502}{\sphinxunichar{2502}}
  \DeclareUnicodeCharacter{2514}{\sphinxunichar{2514}}
  \DeclareUnicodeCharacter{251C}{\sphinxunichar{251C}}
  \DeclareUnicodeCharacter{2572}{\textbackslash}
 \fi
\fi
\usepackage{cmap}
\usepackage[T1]{fontenc}
\usepackage{amsmath,amssymb,amstext}
\usepackage{babel}
\usepackage{times}
\usepackage[Bjarne]{fncychap}
\usepackage{sphinx}

\usepackage{geometry}

% Include hyperref last.
\usepackage{hyperref}
% Fix anchor placement for figures with captions.
\usepackage{hypcap}% it must be loaded after hyperref.
% Set up styles of URL: it should be placed after hyperref.
\urlstyle{same}

\addto\captionsenglish{\renewcommand{\figurename}{Fig.}}
\addto\captionsenglish{\renewcommand{\tablename}{Table}}
\addto\captionsenglish{\renewcommand{\literalblockname}{Listing}}

\addto\captionsenglish{\renewcommand{\literalblockcontinuedname}{continued from previous page}}
\addto\captionsenglish{\renewcommand{\literalblockcontinuesname}{continues on next page}}

\addto\extrasenglish{\def\pageautorefname{page}}

\setcounter{tocdepth}{1}



\title{moha Documentation}
\date{Aug 21, 2018}
\release{1.0.0}
\author{Yilin Zhao}
\newcommand{\sphinxlogo}{\vbox{}}
\renewcommand{\releasename}{Release}
\makeindex

\begin{document}

\maketitle
\sphinxtableofcontents
\phantomsection\label{\detokenize{index::doc}}


MoHa is abbreviation of \sphinxstyleemphasis{Mo}lecular/\sphinxstyleemphasis{Mo}del \sphinxstyleemphasis{Ha}miltonian, it is a quantum chemistry program written in python. MoHa now is not a formal program but a toy for practice coding and help myself to have a better
understanding of quantum chemistry. If you are interested in, Please see more details in the \DUrole{xref,std,std-ref}{Overview}.


\chapter{Contents}
\label{\detokenize{index:contents}}

\section{MoHa Overview}
\label{\detokenize{overview:moha-overview}}\label{\detokenize{overview:overview-rst}}\label{\detokenize{overview::doc}}
Python-based simulations of chemistry framework (PYSCF) is a general-purpose
electronic structure platform designed from the ground up to emphasize code
simplicity, so as to facilitate new method development and enable flexible
computational workflows. The package provides a wide range of tools to support
simulations of finite-size systems, extended systems with periodic boundary
conditions, low-dimensional periodic systems, and custom Hamiltonians, using
mean-field and post-mean-field methods with standard Gaussian basis functions.
To ensure ease of extensibility, PYSCF uses the Python language to implement
almost all of its features, while computationally critical paths are
implemented with heavily optimized C routines. Using this combined Python/C
implementation, the package is as efficient as the best existing C or Fortran-
based quantum chemistry programs.


\subsection{Features}
\label{\detokenize{overview:features}}\begin{itemize}
\item {} 
Interface to integral package \sphinxhref{https://github.com/sunqm/libcint}{Libcint}

\item {} 
Interface to DMRG \sphinxhref{https://github.com/SebWouters/CheMPS2}{CheMPS2}

\item {} 
Interface to DMRG \sphinxhref{https://github.com/sanshar/Block}{Block}

\item {} 
Interface to FCIQMC \sphinxhref{https://github.com/ghb24/NECI\_STABLE}{NECI}

\item {} 
Interface to XC functional library \sphinxhref{https://github.com/dftlibs/xcfun}{XCFun}

\item {} 
Interface to XC functional library \sphinxhref{http://www.tddft.org/programs/octopus/wiki/index.php/Libxc}{Libxc}

\end{itemize}


\section{Installation}
\label{\detokenize{installation:installation}}\label{\detokenize{installation:id1}}\label{\detokenize{installation::doc}}

\section{Molecular System}
\label{\detokenize{user_molecular_system:molecular-system}}\label{\detokenize{user_molecular_system::doc}}
To begin a calculation with MoHa, the first step is to build a Hamiltonian of a
system, either molecular system or model system. In most cases, we need to build a molecular
system

In terms of second quantisation operators, a general Hamiltonian can be written
as
\begin{equation*}
\begin{split}H = - \sum_{ij} t_{ij}\hat{c}^{\dagger}_{i}\hat{c}_{j} + \frac{1}{2} \sum_{ijkl}
V_{ijkl}\hat{c}^{\dagger}_{i}\hat{c}^{\dagger}_{k}\hat{c}_{l}\hat{c}_{j}\end{split}
\end{equation*}
The construction of molecular Hamiltonian usually set up in three steps.
\begin{itemize}
\item {} 
First, construct a molecular geometry.

\item {} 
Second, generate a Gaussian basis set for the molecular.

\item {} 
Finally, compute all kinds of one body terms and two body terms with that basis
to define a Hamiltonian.

\end{itemize}


\subsection{Molecule}
\label{\detokenize{user_molecular_system:molecule}}
Molecule is a system consist with nucleus and electrons. For quantum chemistry
calculation, we will always used the Born-Oppenheimer apporimation, which assumption
that the motion of atomic nuclei and electrons in a molecule can be separated
\begin{equation*}
\begin{split}\Psi_{molecule} = \psi_{electronic} \otimes \psi_{nuclear}\end{split}
\end{equation*}
The module \sphinxcode{\sphinxupquote{molecule}} in MoHa actually only contains imformation of the
nuclear. It has three class:
\begin{itemize}
\item {} 
\sphinxcode{\sphinxupquote{Element}}

\end{itemize}

Represents an element from the periodic table. The following attributes are supported for all elements:
\begin{quote}
\begin{description}
\item[{number}] \leavevmode
The atomic number.

\item[{symbol}] \leavevmode
A string with the symbol of the element.

\item[{name}] \leavevmode
The full element name.

\item[{group}] \leavevmode
The group of the element (not for actinides and lanthanides).

\item[{period}] \leavevmode
The row of the periodic system.

\end{description}
\end{quote}
\begin{itemize}
\item {} 
\sphinxcode{\sphinxupquote{Atom}}

\end{itemize}

Represents an Atom. The following attributes are supported for all atoms:
\begin{quote}
\begin{description}
\item[{element}] \leavevmode
A object of \sphinxcode{\sphinxupquote{Element}} class.

\item[{coordinate}] \leavevmode
The coordinate of the atom object.

\end{description}
\end{quote}
\begin{itemize}
\item {} 
\sphinxcode{\sphinxupquote{Molecule}}

\end{itemize}

Represents an Molecule. The following attributes are supported for all molecule
object:
\begin{quote}
\begin{description}
\item[{title}] \leavevmode
type: string
title of the system

\item[{size}] \leavevmode
type: intager
number of atoms

\item[{symmetry}] \leavevmode
type: string
point group of the molecule

\item[{bond\_length}] \leavevmode\begin{description}
\item[{Calculate the interatomic distances using the expression:}] \leavevmode\begin{equation*}
\begin{split}R_{ij}=\sqrt{(x_i-x_j)^2 + (y_i - y_j)^2 + (z_i - z_j)^2}\end{split}
\end{equation*}
\end{description}

where x, y, and z are Cartesian coordinates and i and j denote atomic indices.

\item[{bond\_angle}] \leavevmode
Calculate all possible bond angles. For example, the angle, \(\phi_{ijk}\), between atoms i-j-k, where j is the central atom is given by:
\begin{quote}
\begin{equation*}
\begin{split}\cos {\phi_{ijk}} = {\bf \vec{e}_{ji} } \cdot {\bf \vec{e}_{jk}}\end{split}
\end{equation*}\end{quote}

where the \(\bf \vec{e}_{ij}\) are unit vectors between the atoms, e.g.,
\begin{quote}
\begin{equation*}
\begin{split}e_{ij}^x = - \left(x_i - x_j \right)/R_{ij},\ \ \ \ \
e_{ij}^y = - \left(y_i - y_j \right)/R_{ij},\ \ \ \ \
e_{ij}^z = - \left(z_i - z_j \right)/R_{ij}\end{split}
\end{equation*}\end{quote}

\item[{out\_of\_plane\_angle}] \leavevmode
Calculate all possible out-of-plane angles. For example, the angle \(\theta_{ijkl}\) for atom i out of the plane containing atoms j-k-l (with k as the central atom, connected to i) is given by:
\begin{quote}
\begin{equation*}
\begin{split}\sin {\theta_{ijkl}} =  \frac{{\bf \vec{e}_{kj} \times \vec{e}_{kl} }}{ \sin {\phi_{jkl}}} \cdot  {\bf \vec{e}_{ki} }\end{split}
\end{equation*}\end{quote}

\end{description}

dihedral\_angle
\begin{quote}

Calculate all possible torsional angles. For example, the torsional angle \(\tau_{ijkl}\) for the atom connectivity i-j-k-l is given by:
\begin{quote}
\begin{equation*}
\begin{split}\cos {\tau_{ijkl}} = \frac{ ({\bf \vec{e}_{ij} \times \vec{e}_{jk} }) \cdot ( {\bf \vec{e}_{jk} \times \vec{e}_{kl} } ) }{  \sin {\phi_{ijk}} \  \ \sin{\phi_{jkl}}}\end{split}
\end{equation*}\end{quote}
\end{quote}
\end{quote}

To be convenient, we can specify the molecular object by load the molecular geometry from file formats.

\fvset{hllines={, ,}}%
\begin{sphinxVerbatim}[commandchars=\\\{\}]
\PYG{n}{mol} \PYG{o}{=} \PYG{n}{IOSystem}\PYG{o}{.}\PYG{n}{from\PYGZus{}file}\PYG{p}{(}\PYG{l+s+s1}{\PYGZsq{}}\PYG{l+s+s1}{h2o.xyz}\PYG{l+s+s1}{\PYGZsq{}}\PYG{p}{)}
\end{sphinxVerbatim}


\subsection{Basis Set}
\label{\detokenize{user_molecular_system:basis-set}}
MoHa supports basis sets consisting of generally contracted Cartesian Gaussian
functions. MoHa is using the same basis set format as NWChem, and the basis sets can be downloaded from the EMSL webpage (\sphinxurl{https://bse.pnl.gov/bse/portal}).

The basis object is a list of list with the follwing structure
\begin{equation*}
\begin{split}[Idx x y z #CONTR CONTR Atomidx]\end{split}
\end{equation*}
Idx is the AO index. Atomidx, is the index of the associated atom. \#CONTR contains the number of primitive functions in the contracted.

CONTR contains all the information about the primitive functions and have the form:
\begin{equation*}
\begin{split}[N ζ c l m n]\end{split}
\end{equation*}
Inside the integral call, the basisset file is reconstructed into three different arrays, containing the basisset information. The first one is basisidx that have the following form:
\begin{equation*}
\begin{split}[#primitives\ \ \ loopstartidx]\end{split}
\end{equation*}
It thus contains the number of primitives in each basisfunction, and what start index it have for loop inside the integral code.

The second array is basisint, that have the following forms:
\begin{equation*}
\begin{split}[l m n]
[l m n atomidx]\end{split}
\end{equation*}
The first one is for regular integrals and the second one is for derivatives. Both contains all the angular momentum quantum numbers, and the derivative also contains the atom index (used in derivative of VNe).

The last array is basisfloat and have the following forms:
\begin{equation*}
\begin{split}[N ζ c x y z]
[N ζ c x y z N_{x,+} N_{x,−} N_{y,+} N_{y,−} N_{z,+} N_{z,−}]\end{split}
\end{equation*}
basisfloat contains the normalization constants, Gaussian exponent and prefacor and the coordinates of the atoms. The second one is again for the derivatives, it contains normalization constants of the differentiated primitives.


\subsection{Hamilonian}
\label{\detokenize{user_molecular_system:hamilonian}}

\subsection{Molecular Integrals}
\label{\detokenize{user_molecular_system:molecular-integrals}}
Contains the information about how the integrals are calculated. In the equations in this section the following definitions is used.
\begin{equation*}
\begin{split}p=a+b
μ=aba+b
Px=aAx+bBxp
XAB=Ax−Bx\end{split}
\end{equation*}
Here a and b are Gaussian exponent factors. Ax and Bx are the position of the Gaussians in one dimension. Further the basisset functions is of Gaussian kind and given as:
\begin{equation*}
\begin{split}ϕA(r)=N(x−Ax)l(y−Ay)m(z−Az)nexp(−ζ(r⃗ −A⃗ )2)\end{split}
\end{equation*}
with a normalization constant given as:
\begin{equation*}
\begin{split}N=(2απ)3/4[(8α)l+m+nl!m!n!(2l)!(2m)!(2n)!]\end{split}
\end{equation*}

\subsubsection{Boys Function}
\label{\detokenize{user_molecular_system:boys-function}}
The Boys function is given as:
.. math:

\fvset{hllines={, ,}}%
\begin{sphinxVerbatim}[commandchars=\\\{\}]
Fn(x)=∫10exp(−xt2)t2ndt
\end{sphinxVerbatim}

FUNCTION:

MIcython.boys(m,T)
return value
Input:

m, subscript of the Boys function
T, argument of the Boys function
Output:

value, value corrosponding to given m and T


\subsubsection{Expansion coefficients}
\label{\detokenize{user_molecular_system:expansion-coefficients}}
The expansion coefficient is found by the following recurrence relation:

Ei,jt=0,t\textless{}0ort\textgreater{}i+j
Ei+1,jt=12pEi,jt−1+XPAEi,jt+(t+1)Ei,jt+1
Ei,j+1t=12pEi,jt−1+XPBEi,jt+(t+1)Ei,jt+1
With the boundary condition that:

E0,00=exp(−pX2AB)
FUNCTION:

MolecularIntegrals.E(i,j,t,Qx,a,b,XPA,XPB,XAB)
return val
Input:

i, input values
j, input values
t, input values
Qx, input values
a, input values
b, input values
XPA, input values
XPB, input values
XAB, input values
Output:

val, value corrosponding to the given input


\subsubsection{Overlap}
\label{\detokenize{user_molecular_system:overlap}}
The overlap integrals are solved by the following recurrence relation:
\begin{equation*}
\begin{split}Si+1,j=XPASij+12p(iSi−1,j+jSi,j−1)
Si,j+1=XPBSij+12p(iSi−1,j+jSi,j−1)\end{split}
\end{equation*}
With the boundary condition that:
\begin{equation*}
\begin{split}S00=πp‾‾√exp(−μX2AB)\end{split}
\end{equation*}
FUNCTION:

MolecularIntegrals.Overlap(a, b, la, lb, Ax, Bx)
return Sij
Input:

a, Gaussian exponent factor
b, Gaussian exponent factor
la, angular momentum quantum number
lb, angular momentum quantum number
Ax, position along one axis
Bx, position along one axis
Output:

Sij, non-normalized overlap element in one dimension


\subsubsection{Kinetic energy}
\label{\detokenize{user_molecular_system:kinetic-energy}}
The kinetic energy integrals are solved by the following recurrence relation:
\begin{equation*}
\begin{split}Ti+1,j=XPATi,j+12p(iTi−1,j+jTi,j−1)+bp(2aSi+1,j−iSi−1,j)
Ti,j+1=XPBTi,j+12p(iTi−1,j+jTi,j−1)+ap(2bSi,j+1−iSi,j−1)\end{split}
\end{equation*}
With the boundary condition that:
\begin{equation*}
\begin{split}T00=[a−2a2(X2PA+12p)]S00\end{split}
\end{equation*}
FUNCTION:

Kin(a, b, Ax, Ay, Az, Bx, By, Bz, la, lb, ma, mb, na, nb, N1, N2, c1, c2)
return Tij, Sij
Input:

a, Gaussian exponent factor
b, Gaussian exponent factor
Ax, position along the x-axis
Bx, position along the x-axis
Ay, position along the y-axis
By, position along the y-axis
Az, position along the z-axis
Bz, position along the z-axis
la, angular momentum quantum number
lb, angular momentum quantum number
ma, angular momentum quantum number
mb, angular momentum quantum number
na, angular momentum quantum number
nb, angular momentum quantum number
N1, normalization constant
N2, normalization constant
c1, Gaussian prefactor
c2, Gaussian prefactor
Output:

Tij, normalized kinetic energy matrix element
Sij, normalized overlap matrix element


\subsubsection{Electron-nuclear attraction}
\label{\detokenize{user_molecular_system:electron-nuclear-attraction}}
The electron-nuclear interaction integral is given as:

V000ijklmn=2\(\pi\)p∑ti+jEijt∑uk+lEklu∑vm+nEmnvRtuv
FUNCTION:

MolecularIntegrals.elnuc(P, p, l1, l2, m1, m2, n1, n2, N1, N2, c1, c2, Zc, Ex, Ey, Ez, R1)
return Vij
Input:

P, Gaussian product
p, exponent from Guassian product
l1, angular momentum quantum number
l2, angular momentum quantum number
m1, angular momentum quantum number
n1, angular momentum quantum number
n2, angular momentum quantum number
N1, normalization constant
N2, normalization constant
c1, Gaussian prefactor
c2, Gaussian prefactor
Zc, Nuclear charge
Ex, expansion coefficients
Ey, expansion coefficients
Ez, expansion coefficients
R1, hermite coulomb integrals
Output:

Vij, normalized electron-nuclei attraction matrix element


\subsubsection{Electron-electron repulsion}
\label{\detokenize{user_molecular_system:electron-electron-repulsion}}
The electron-electron repulsion integral is calculated as:

gabcd=∑tl1+l2Eabt∑um1+m2Eabu∑vn1+n2Eabv∑\(\tau\)l3+l4Ecd\(\tau\)∑\(\nu\)m3+m4Ecd\(\nu\)∑ϕn3+n4Ecdϕ(−1)\(\tau\)+\(\nu\)+ϕ2\(\pi\)5/2pqp+q‾‾‾‾‾√Rt+\(\tau\),u+\(\nu\),v+ϕ(\(\alpha\),RPQ)
FUNCTION:

MIcython.elelrep(p, q, l1, l2, l3, l4, m1, m2, m3, m4, n1, n2, n3, n4, N1, N2, N3, N4, c1, c2, c3, c4, E1, E2, E3, E4, E5, E6, Rpre)
return Veeijkl
Input:

p, Gaussian exponent factor from Gaussian product
q, Gaussian exponent factor from Gaussian product
l1, angular momentum quantum number
l2, angular momentum quantum number
l3, angular momentum quantum number
l4, angular momentum quantum number
m1, angular momentum quantum number
m2, angular momentum quantum number
m3, angular momentum quantum number
m4, angular momentum quantum number
n1, angular momentum quantum number
n2, angular momentum quantum number
n3, angular momentum quantum number
n4, angular momentum quantum number
N1, normalization constant
N2, normalization constant
N3, normalization constant
N4, normalization constant
c1, Gaussian prefactor
c2, Gaussian prefactor
c3, Gaussian prefactor
c4, Gaussian prefactor
E1, expansion coefficient
E2, expansion coefficient
E3, expansion coefficient
E4, expansion coefficient
E5, expansion coefficient
E6, expansion coefficient
Rpre, hermite coulomb integral
Output:

Veeijkl, normalized electron-electron repulsion matrix element


\subsection{Build Hamilton}
\label{\detokenize{user_molecular_system:build-hamilton}}
To build a Hamiltonian object, MoHa can load the molecular geometry and and basis from file
format.

\fvset{hllines={, ,}}%
\begin{sphinxVerbatim}[commandchars=\\\{\}]
\PYG{n}{mol}\PYG{p}{,}\PYG{n}{orbs} \PYG{o}{=} \PYG{n}{IOSystem}\PYG{o}{.}\PYG{n}{from\PYGZus{}file}\PYG{p}{(}\PYG{l+s+s1}{\PYGZsq{}}\PYG{l+s+s1}{h2o.xyz}\PYG{l+s+s1}{\PYGZsq{}}\PYG{p}{,}\PYG{l+s+s1}{\PYGZsq{}}\PYG{l+s+s1}{sto\PYGZhy{}3g.nwchem}\PYG{l+s+s1}{\PYGZsq{}}\PYG{p}{)}
\PYG{n}{ham} \PYG{o}{=} \PYG{n}{Hamiltonian}\PYG{o}{.}\PYG{n}{build}\PYG{p}{(}\PYG{n}{mol}\PYG{p}{,}\PYG{n}{orbs}\PYG{p}{)}
\end{sphinxVerbatim}


\section{Model System}
\label{\detokenize{user_model_system:model-system}}\label{\detokenize{user_model_system::doc}}
In terms of second quantisation operators, a general Hamiltonian can be written as
\begin{equation*}
\begin{split}H = - \sum_{ij} t_{ij}\hat{c}^{\dagger}_{i}\hat{c}_{j} + \frac{1}{2} \sum_{ijkl}
V_{ijkl}\hat{c}^{\dagger}_{i}\hat{c}^{\dagger}_{k}\hat{c}_{l}\hat{c}_{j}\end{split}
\end{equation*}
where \(c_{i, \sigma}\) is the destruction operator for an electron at site \(i\)
and spin \(\sigma\), \(c^{\dagger}_{i, \sigma}\) is the creation operator for an
electron at site \(i\) and spin \(\sigma\).

Model Hamiltonians geared to simulate the key physics of notoriously complicated
complete Hamiltonians of large-scale interacting systems.


\subsection{Lattice}
\label{\detokenize{user_model_system:lattice}}
Lattice is an periodic array of discrete points in space generated by a set of
discrete translation operations described by:
\begin{equation*}
\begin{split}\vec{R} = n_1\vec{a}_1 + n_2\vec{a}_2 + n_3\vec{a}_3\end{split}
\end{equation*}
where \(n_i\) are any integers and \(\vec{a}_i\) are known as the primitive
vectors which lie in different directions and span the lattice. The discrete points
can be atoms, ions, electronic site, spin one-half site, spinless fermion site, etc.

Generaly speaking, to build a The \sphinxcode{\sphinxupquote{Lattice}} object, we need first define a
\sphinxcode{\sphinxupquote{Cell}} object. \sphinxcode{\sphinxupquote{Cell}} has attribute of dimension and three primitive
vectors \(\vec{a}_1\) \(\vec{a}_2\) and \(\vec{a}_3\). To initialize a
cell:

\fvset{hllines={, ,}}%
\begin{sphinxVerbatim}[commandchars=\\\{\}]
\PYG{n}{cell} \PYG{o}{=} \PYG{n}{Cell}\PYG{p}{(}\PYG{l+m+mi}{1}\PYG{p}{,}\PYG{p}{[}\PYG{n}{d}\PYG{p}{,}\PYG{l+m+mf}{0.}\PYG{p}{,}\PYG{l+m+mf}{0.}\PYG{p}{]}\PYG{p}{,}\PYG{p}{[}\PYG{l+m+mf}{0.}\PYG{p}{,}\PYG{l+m+mf}{0.}\PYG{p}{,}\PYG{l+m+mf}{0.}\PYG{p}{]}\PYG{p}{,}\PYG{p}{[}\PYG{l+m+mf}{0.}\PYG{p}{,}\PYG{l+m+mf}{0.}\PYG{p}{,}\PYG{l+m+mf}{0.}\PYG{p}{]}\PYG{p}{)}
\end{sphinxVerbatim}

Then we can add arbitrary sites to this cell by assign coordinate and label of the
site:

\fvset{hllines={, ,}}%
\begin{sphinxVerbatim}[commandchars=\\\{\}]
\PYG{n}{cell}\PYG{o}{.}\PYG{n}{add\PYGZus{}site}\PYG{p}{(}\PYG{n}{LatticeSite}\PYG{p}{(}\PYG{p}{[}\PYG{l+m+mf}{0.}\PYG{p}{,}\PYG{l+m+mf}{0.}\PYG{p}{,}\PYG{l+m+mf}{0.}\PYG{p}{]}\PYG{p}{,}\PYG{l+s+s1}{\PYGZsq{}}\PYG{l+s+s1}{A}\PYG{l+s+s1}{\PYGZsq{}}\PYG{p}{)}\PYG{p}{)}
\end{sphinxVerbatim}

Then we add bonds to this cell by assign the integer coordinate of first cell, index
of first site in first cell and the integer coordinate of second cell, index of the
second site in second cell:

\fvset{hllines={, ,}}%
\begin{sphinxVerbatim}[commandchars=\\\{\}]
\PYG{n}{cell}\PYG{o}{.}\PYG{n}{add\PYGZus{}bond}\PYG{p}{(}\PYG{n}{LatticeBond}\PYG{p}{(}\PYG{p}{[}\PYG{l+m+mi}{0}\PYG{p}{,}\PYG{l+m+mi}{0}\PYG{p}{,}\PYG{l+m+mi}{0}\PYG{p}{]}\PYG{p}{,}\PYG{l+m+mi}{0}\PYG{p}{,}\PYG{p}{[}\PYG{l+m+mi}{1}\PYG{p}{,}\PYG{l+m+mi}{0}\PYG{p}{,}\PYG{l+m+mi}{0}\PYG{p}{]}\PYG{p}{,}\PYG{l+m+mi}{0}\PYG{p}{)}\PYG{p}{)}
\end{sphinxVerbatim}

Finally complete it by assign the cell and shpe to the lattice to \sphinxcode{\sphinxupquote{Lattice}}:

\fvset{hllines={, ,}}%
\begin{sphinxVerbatim}[commandchars=\\\{\}]
\PYG{n}{lattice} \PYG{o}{=} \PYG{n}{Lattice}\PYG{p}{(}\PYG{n}{cell}\PYG{p}{,}\PYG{p}{[}\PYG{l+m+mi}{4}\PYG{p}{,}\PYG{l+m+mi}{1}\PYG{p}{,}\PYG{l+m+mi}{1}\PYG{p}{]}\PYG{p}{)}
\end{sphinxVerbatim}

It may not be immediately obvious what this code does. Fortunately, \sphinxcode{\sphinxupquote{Lattice}} objects
have a convenient \sphinxcode{\sphinxupquote{Lattice.plot()}} method to easily visualize the constructed lattice.


\subsubsection{Linear lattice}
\label{\detokenize{user_model_system:linear-lattice}}
Starting from the basics, we’ll build a simple linear lattice with only one site.

\def\sphinxLiteralBlockLabel{\label{\detokenize{user_model_system:id1}}}
\sphinxSetupCaptionForVerbatim{/data/examples/modelsystem/linear.py}
\fvset{hllines={, ,}}%
\begin{sphinxVerbatim}[commandchars=\\\{\}]
\PYG{k+kn}{from} \PYG{n+nn}{moha} \PYG{k}{import} \PYG{o}{*}

\PYG{n}{d} \PYG{o}{=} \PYG{l+m+mf}{1.0} \PYG{c+c1}{\PYGZsh{} unit cell length}

\PYG{k}{def} \PYG{n+nf}{linear}\PYG{p}{(}\PYG{n}{d}\PYG{p}{)}\PYG{p}{:}
    \PYG{c+c1}{\PYGZsh{}define a 1D linear lattice cell with vectors a1 a2 and a3}
    \PYG{n}{cell} \PYG{o}{=} \PYG{n}{Cell}\PYG{p}{(}\PYG{l+m+mi}{1}\PYG{p}{,}\PYG{p}{[}\PYG{n}{d}\PYG{p}{,}\PYG{l+m+mf}{0.}\PYG{p}{,}\PYG{l+m+mf}{0.}\PYG{p}{]}\PYG{p}{,}\PYG{p}{[}\PYG{l+m+mf}{0.}\PYG{p}{,}\PYG{l+m+mf}{0.}\PYG{p}{,}\PYG{l+m+mf}{0.}\PYG{p}{]}\PYG{p}{,}\PYG{p}{[}\PYG{l+m+mf}{0.}\PYG{p}{,}\PYG{l+m+mf}{0.}\PYG{p}{,}\PYG{l+m+mf}{0.}\PYG{p}{]}\PYG{p}{)}
    \PYG{c+c1}{\PYGZsh{}add a site labeled \PYGZsq{}A\PYGZsq{} at positon [0.,0.,0.]}
    \PYG{n}{cell}\PYG{o}{.}\PYG{n}{add\PYGZus{}site}\PYG{p}{(}\PYG{n}{LatticeSite}\PYG{p}{(}\PYG{p}{[}\PYG{l+m+mf}{0.}\PYG{p}{,}\PYG{l+m+mf}{0.}\PYG{p}{,}\PYG{l+m+mf}{0.}\PYG{p}{]}\PYG{p}{,}\PYG{l+s+s1}{\PYGZsq{}}\PYG{l+s+s1}{A}\PYG{l+s+s1}{\PYGZsq{}}\PYG{p}{)}\PYG{p}{)}
    \PYG{c+c1}{\PYGZsh{}add a bond from first site in cell [0,0,0] to first site in cell [1,0,0]}
    \PYG{n}{cell}\PYG{o}{.}\PYG{n}{add\PYGZus{}bond}\PYG{p}{(}\PYG{n}{LatticeBond}\PYG{p}{(}\PYG{p}{[}\PYG{l+m+mi}{0}\PYG{p}{,}\PYG{l+m+mi}{0}\PYG{p}{,}\PYG{l+m+mi}{0}\PYG{p}{]}\PYG{p}{,}\PYG{l+m+mi}{0}\PYG{p}{,}\PYG{p}{[}\PYG{l+m+mi}{1}\PYG{p}{,}\PYG{l+m+mi}{0}\PYG{p}{,}\PYG{l+m+mi}{0}\PYG{p}{]}\PYG{p}{,}\PYG{l+m+mi}{0}\PYG{p}{)}\PYG{p}{)}
    \PYG{c+c1}{\PYGZsh{}buld a lattice of shape [4,1,1] with cell we defined}
    \PYG{n}{lattice} \PYG{o}{=} \PYG{n}{Lattice}\PYG{p}{(}\PYG{n}{cell}\PYG{p}{,}\PYG{p}{[}\PYG{l+m+mi}{4}\PYG{p}{,}\PYG{l+m+mi}{1}\PYG{p}{,}\PYG{l+m+mi}{1}\PYG{p}{]}\PYG{p}{)}
    \PYG{k}{return} \PYG{n}{lattice}

\PYG{n}{lattice} \PYG{o}{=} \PYG{n}{linear}\PYG{p}{(}\PYG{n}{d}\PYG{p}{)}
\PYG{c+c1}{\PYGZsh{}plot the lattice we constructed}
\PYG{n}{lattice}\PYG{o}{.}\PYG{n}{plot}\PYG{p}{(}\PYG{p}{)} 
\end{sphinxVerbatim}

Visualize the lattice by \sphinxcode{\sphinxupquote{Lattice.plot()}} method.

\begin{figure}[htbp]
\centering

\noindent\sphinxincludegraphics[scale=1.0]{{linear}.png}
\end{figure}


\subsubsection{Two sites linear lattice}
\label{\detokenize{user_model_system:two-sites-linear-lattice}}
The next example is also linear lattice but a slightly more complicated, the cell of the
lattice is consist of two different sites.

\def\sphinxLiteralBlockLabel{\label{\detokenize{user_model_system:id2}}}
\sphinxSetupCaptionForVerbatim{/data/examples/modelsystem/two\_sites\_linear.py}
\fvset{hllines={, ,}}%
\begin{sphinxVerbatim}[commandchars=\\\{\}]
\PYG{k+kn}{from} \PYG{n+nn}{moha} \PYG{k}{import} \PYG{o}{*}

\PYG{n}{d} \PYG{o}{=} \PYG{l+m+mf}{1.0}     \PYG{c+c1}{\PYGZsh{} unit cell length}
\PYG{n}{d\PYGZus{}ab} \PYG{o}{=} \PYG{l+m+mf}{0.2}  \PYG{c+c1}{\PYGZsh{} distance between site A and B}

\PYG{k}{def} \PYG{n+nf}{linear}\PYG{p}{(}\PYG{n}{d}\PYG{p}{,}\PYG{n}{d\PYGZus{}ab}\PYG{p}{)}\PYG{p}{:}
    \PYG{c+c1}{\PYGZsh{}define a 1D linear lattice cell with vectors a1 a2 and a3}
    \PYG{n}{cell} \PYG{o}{=} \PYG{n}{Cell}\PYG{p}{(}\PYG{l+m+mi}{1}\PYG{p}{,}\PYG{p}{[}\PYG{n}{d}\PYG{p}{,} \PYG{l+m+mf}{0.}\PYG{p}{,} \PYG{l+m+mf}{0.}\PYG{p}{]}\PYG{p}{,}\PYG{p}{[}\PYG{l+m+mf}{0.}\PYG{p}{,} \PYG{l+m+mf}{0.}\PYG{p}{,} \PYG{l+m+mf}{0.}\PYG{p}{]}\PYG{p}{,}\PYG{p}{[}\PYG{l+m+mf}{0.}\PYG{p}{,} \PYG{l+m+mf}{0.}\PYG{p}{,} \PYG{l+m+mf}{0.}\PYG{p}{]}\PYG{p}{)}
    \PYG{c+c1}{\PYGZsh{}add a site labeled \PYGZsq{}A\PYGZsq{} at positon [0., 0., 0.]}
    \PYG{n}{cell}\PYG{o}{.}\PYG{n}{add\PYGZus{}site}\PYG{p}{(}\PYG{n}{LatticeSite}\PYG{p}{(}\PYG{p}{[}\PYG{l+m+mf}{0.}\PYG{p}{,} \PYG{l+m+mf}{0.}\PYG{p}{,} \PYG{l+m+mf}{0.}\PYG{p}{]}\PYG{p}{,} \PYG{l+s+s1}{\PYGZsq{}}\PYG{l+s+s1}{A}\PYG{l+s+s1}{\PYGZsq{}}\PYG{p}{)}\PYG{p}{)}
    \PYG{c+c1}{\PYGZsh{}add a site labeled \PYGZsq{}B\PYGZsq{} at positon [d\PYGZus{}ab, 0., 0.]}
    \PYG{n}{cell}\PYG{o}{.}\PYG{n}{add\PYGZus{}site}\PYG{p}{(}\PYG{n}{LatticeSite}\PYG{p}{(}\PYG{p}{[}\PYG{n}{d\PYGZus{}ab}\PYG{p}{,} \PYG{l+m+mf}{0.}\PYG{p}{,} \PYG{l+m+mf}{0.}\PYG{p}{]}\PYG{p}{,}\PYG{l+s+s1}{\PYGZsq{}}\PYG{l+s+s1}{B}\PYG{l+s+s1}{\PYGZsq{}}\PYG{p}{)}\PYG{p}{)}
    \PYG{c+c1}{\PYGZsh{}add a bond from first site in cell [0,0,0] to second site in cell [0,0,0]}
    \PYG{n}{cell}\PYG{o}{.}\PYG{n}{add\PYGZus{}bond}\PYG{p}{(}\PYG{n}{LatticeBond}\PYG{p}{(}\PYG{p}{[}\PYG{l+m+mi}{0}\PYG{p}{,}\PYG{l+m+mi}{0}\PYG{p}{,}\PYG{l+m+mi}{0}\PYG{p}{]}\PYG{p}{,}\PYG{l+m+mi}{0}\PYG{p}{,}\PYG{p}{[}\PYG{l+m+mi}{0}\PYG{p}{,}\PYG{l+m+mi}{0}\PYG{p}{,}\PYG{l+m+mi}{0}\PYG{p}{]}\PYG{p}{,}\PYG{l+m+mi}{1}\PYG{p}{)}\PYG{p}{)}
    \PYG{c+c1}{\PYGZsh{}add a bond from second site in cell [0,0,0] to first site in cell [1,0,0]}
    \PYG{n}{cell}\PYG{o}{.}\PYG{n}{add\PYGZus{}bond}\PYG{p}{(}\PYG{n}{LatticeBond}\PYG{p}{(}\PYG{p}{[}\PYG{l+m+mi}{0}\PYG{p}{,}\PYG{l+m+mi}{0}\PYG{p}{,}\PYG{l+m+mi}{0}\PYG{p}{]}\PYG{p}{,}\PYG{l+m+mi}{1}\PYG{p}{,}\PYG{p}{[}\PYG{l+m+mi}{1}\PYG{p}{,}\PYG{l+m+mi}{0}\PYG{p}{,}\PYG{l+m+mi}{0}\PYG{p}{]}\PYG{p}{,}\PYG{l+m+mi}{0}\PYG{p}{)}\PYG{p}{)}
    \PYG{c+c1}{\PYGZsh{}buld a lattice of shape [4,1,1] with cell we defined}
    \PYG{n}{lattice} \PYG{o}{=} \PYG{n}{Lattice}\PYG{p}{(}\PYG{n}{cell}\PYG{p}{,}\PYG{p}{[}\PYG{l+m+mi}{4}\PYG{p}{,}\PYG{l+m+mi}{1}\PYG{p}{,}\PYG{l+m+mi}{1}\PYG{p}{]}\PYG{p}{)}
    \PYG{k}{return} \PYG{n}{lattice}

\end{sphinxVerbatim}

Visualize the lattice by \sphinxcode{\sphinxupquote{Lattice.plot()}} method.

\begin{figure}[htbp]
\centering

\noindent\sphinxincludegraphics[scale=1.0]{{two_sites_linear}.png}
\end{figure}


\subsubsection{Square lattice}
\label{\detokenize{user_model_system:square-lattice}}
From 1D to 2D, the most basic example square lattice.

\def\sphinxLiteralBlockLabel{\label{\detokenize{user_model_system:id3}}}
\sphinxSetupCaptionForVerbatim{/data/examples/modelsystem/square.py}
\fvset{hllines={, ,}}%
\begin{sphinxVerbatim}[commandchars=\\\{\}]
\PYG{k+kn}{from} \PYG{n+nn}{moha} \PYG{k}{import} \PYG{o}{*}

\PYG{n}{a} \PYG{o}{=} \PYG{l+m+mf}{1.0} \PYG{c+c1}{\PYGZsh{}unit cell length}

\PYG{k}{def} \PYG{n+nf}{square}\PYG{p}{(}\PYG{n}{a}\PYG{p}{)}\PYG{p}{:}
    \PYG{c+c1}{\PYGZsh{}define a 2D square lattice cell with vectors a1 a2 and a3}
    \PYG{n}{cell} \PYG{o}{=} \PYG{n}{Cell}\PYG{p}{(}\PYG{l+m+mi}{2}\PYG{p}{,}\PYG{p}{[}\PYG{n}{a}\PYG{p}{,} \PYG{l+m+mf}{0.}\PYG{p}{,} \PYG{l+m+mf}{0.}\PYG{p}{]}\PYG{p}{,}\PYG{p}{[}\PYG{l+m+mf}{0.}\PYG{p}{,} \PYG{n}{a}\PYG{p}{,} \PYG{l+m+mf}{0.}\PYG{p}{]}\PYG{p}{,}\PYG{p}{[}\PYG{l+m+mf}{0.}\PYG{p}{,} \PYG{l+m+mf}{0.}\PYG{p}{,} \PYG{l+m+mf}{0.}\PYG{p}{]}\PYG{p}{)}
    \PYG{c+c1}{\PYGZsh{}add a site labeled \PYGZsq{}A\PYGZsq{} at positon [0.,0.,0.]}
    \PYG{n}{cell}\PYG{o}{.}\PYG{n}{add\PYGZus{}site}\PYG{p}{(}\PYG{n}{LatticeSite}\PYG{p}{(}\PYG{p}{[}\PYG{l+m+mf}{0.}\PYG{p}{,}\PYG{l+m+mf}{0.}\PYG{p}{,}\PYG{l+m+mf}{0.}\PYG{p}{]}\PYG{p}{,}\PYG{l+s+s1}{\PYGZsq{}}\PYG{l+s+s1}{A}\PYG{l+s+s1}{\PYGZsq{}}\PYG{p}{)}\PYG{p}{)}
    \PYG{c+c1}{\PYGZsh{}add a bond from first site in cell [0,0,0] to first site in cell [1,0,0]}
    \PYG{n}{cell}\PYG{o}{.}\PYG{n}{add\PYGZus{}bond}\PYG{p}{(}\PYG{n}{LatticeBond}\PYG{p}{(}\PYG{p}{[}\PYG{l+m+mi}{0}\PYG{p}{,}\PYG{l+m+mi}{0}\PYG{p}{,}\PYG{l+m+mi}{0}\PYG{p}{]}\PYG{p}{,}\PYG{l+m+mi}{0}\PYG{p}{,}\PYG{p}{[}\PYG{l+m+mi}{1}\PYG{p}{,}\PYG{l+m+mi}{0}\PYG{p}{,}\PYG{l+m+mi}{0}\PYG{p}{]}\PYG{p}{,}\PYG{l+m+mi}{0}\PYG{p}{)}\PYG{p}{)}
    \PYG{c+c1}{\PYGZsh{}add a bond from first site in cell [0,0,0] to first site in cell [0,1,0]}
    \PYG{n}{cell}\PYG{o}{.}\PYG{n}{add\PYGZus{}bond}\PYG{p}{(}\PYG{n}{LatticeBond}\PYG{p}{(}\PYG{p}{[}\PYG{l+m+mi}{0}\PYG{p}{,}\PYG{l+m+mi}{0}\PYG{p}{,}\PYG{l+m+mi}{0}\PYG{p}{]}\PYG{p}{,}\PYG{l+m+mi}{0}\PYG{p}{,}\PYG{p}{[}\PYG{l+m+mi}{0}\PYG{p}{,}\PYG{l+m+mi}{1}\PYG{p}{,}\PYG{l+m+mi}{0}\PYG{p}{]}\PYG{p}{,}\PYG{l+m+mi}{0}\PYG{p}{)}\PYG{p}{)}
    \PYG{c+c1}{\PYGZsh{}buld a lattice of shape [4,4,1] with cell we defined}
    \PYG{n}{lattice} \PYG{o}{=} \PYG{n}{Lattice}\PYG{p}{(}\PYG{n}{cell}\PYG{p}{,}\PYG{p}{[}\PYG{l+m+mi}{4}\PYG{p}{,}\PYG{l+m+mi}{4}\PYG{p}{,}\PYG{l+m+mi}{1}\PYG{p}{]}\PYG{p}{)}
    \PYG{k}{return} \PYG{n}{lattice}

\PYG{n}{lattice} \PYG{o}{=} \PYG{n}{square}\PYG{p}{(}\PYG{n}{a}\PYG{p}{)}
\PYG{c+c1}{\PYGZsh{}plot the lattice we constructed}
\PYG{n}{lattice}\PYG{o}{.}\PYG{n}{plot}\PYG{p}{(}\PYG{p}{)}
\end{sphinxVerbatim}

Visualize the lattice by \sphinxcode{\sphinxupquote{Lattice.plot()}} method.

\begin{figure}[htbp]
\centering

\noindent\sphinxincludegraphics[scale=1.0]{{square}.png}
\end{figure}


\subsubsection{Graphene lattice}
\label{\detokenize{user_model_system:graphene-lattice}}
Graphene lattice is a more general example:
\begin{itemize}
\item {} 
It is two dimension.

\item {} 
The cell of graphene contain two sites.

\item {} 
The primitive vectors of the cell are non-orthogonal.

\end{itemize}

\def\sphinxLiteralBlockLabel{\label{\detokenize{user_model_system:id4}}}
\sphinxSetupCaptionForVerbatim{/data/examples/modelsystem/graphene.py}
\fvset{hllines={, ,}}%
\begin{sphinxVerbatim}[commandchars=\\\{\}]
\PYG{k+kn}{from} \PYG{n+nn}{moha} \PYG{k}{import} \PYG{o}{*}
\PYG{k+kn}{from} \PYG{n+nn}{math} \PYG{k}{import} \PYG{n}{sqrt}

\PYG{n}{a} \PYG{o}{=} \PYG{l+m+mf}{0.24595}     \PYG{c+c1}{\PYGZsh{} unit cell length}
\PYG{n}{a\PYGZus{}cc} \PYG{o}{=} \PYG{l+m+mf}{0.142}    \PYG{c+c1}{\PYGZsh{} carbon carbon distance}

\PYG{k}{def} \PYG{n+nf}{graphene}\PYG{p}{(}\PYG{n}{a}\PYG{p}{,}\PYG{n}{a\PYGZus{}cc}\PYG{p}{)}\PYG{p}{:}


    \PYG{c+c1}{\PYGZsh{}define a 2D graphene lattice cell with vectors a1 a2 and a3}
    \PYG{n}{cell} \PYG{o}{=} \PYG{n}{Cell}\PYG{p}{(}\PYG{l+m+mi}{2}\PYG{p}{,}\PYG{p}{[}\PYG{p}{[}\PYG{n}{a}\PYG{p}{,}\PYG{l+m+mf}{0.}\PYG{p}{,}\PYG{l+m+mf}{0.}\PYG{p}{]}\PYG{p}{,}\PYG{p}{[}\PYG{n}{a}\PYG{o}{/}\PYG{l+m+mi}{2}\PYG{p}{,}\PYG{n}{a}\PYG{o}{/}\PYG{l+m+mi}{2}\PYG{o}{*}\PYG{n}{sqrt}\PYG{p}{(}\PYG{l+m+mi}{3}\PYG{p}{)}\PYG{p}{,}\PYG{l+m+mf}{0.}\PYG{p}{]}\PYG{p}{,}\PYG{p}{[}\PYG{l+m+mf}{0.}\PYG{p}{,}\PYG{l+m+mf}{0.}\PYG{p}{,}\PYG{l+m+mf}{0.}\PYG{p}{]}\PYG{p}{]}\PYG{p}{)}


    \PYG{c+c1}{\PYGZsh{}add a site labeled \PYGZsq{}A\PYGZsq{} at positon [0., \PYGZhy{}a\PYGZus{}cc/2., 0.]}
    \PYG{n}{cell}\PYG{o}{.}\PYG{n}{add\PYGZus{}site}\PYG{p}{(}\PYG{n}{LatticeSite}\PYG{p}{(}\PYG{p}{[}\PYG{l+m+mf}{0.}\PYG{p}{,} \PYG{o}{\PYGZhy{}}\PYG{n}{a\PYGZus{}cc}\PYG{o}{/}\PYG{l+m+mf}{2.}\PYG{p}{,} \PYG{l+m+mf}{0.}\PYG{p}{]}\PYG{p}{,}\PYG{l+s+s1}{\PYGZsq{}}\PYG{l+s+s1}{A}\PYG{l+s+s1}{\PYGZsq{}}\PYG{p}{)}\PYG{p}{)}
    \PYG{c+c1}{\PYGZsh{}add a site labeled \PYGZsq{}B\PYGZsq{} at positon [0., a\PYGZus{}cc/2., 0.]}
    \PYG{n}{cell}\PYG{o}{.}\PYG{n}{add\PYGZus{}site}\PYG{p}{(}\PYG{n}{LatticeSite}\PYG{p}{(}\PYG{p}{[}\PYG{l+m+mf}{0.}\PYG{p}{,} \PYG{n}{a\PYGZus{}cc}\PYG{o}{/}\PYG{l+m+mf}{2.}\PYG{p}{,} \PYG{l+m+mf}{0.}\PYG{p}{]}\PYG{p}{,}\PYG{l+s+s1}{\PYGZsq{}}\PYG{l+s+s1}{B}\PYG{l+s+s1}{\PYGZsq{}}\PYG{p}{)}\PYG{p}{)}

    \PYG{c+c1}{\PYGZsh{}add a bond from first site in cell [0,0,0] to second site in cell [0,0,0]}
    \PYG{n}{cell}\PYG{o}{.}\PYG{n}{add\PYGZus{}bond}\PYG{p}{(}\PYG{n}{LatticeBond}\PYG{p}{(}\PYG{p}{[}\PYG{l+m+mi}{0}\PYG{p}{,}\PYG{l+m+mi}{0}\PYG{p}{,}\PYG{l+m+mi}{0}\PYG{p}{]}\PYG{p}{,}\PYG{l+m+mi}{0}\PYG{p}{,}\PYG{p}{[}\PYG{l+m+mi}{0}\PYG{p}{,}\PYG{l+m+mi}{0}\PYG{p}{,}\PYG{l+m+mi}{0}\PYG{p}{]}\PYG{p}{,}\PYG{l+m+mi}{1}\PYG{p}{)}\PYG{p}{)}
    \PYG{c+c1}{\PYGZsh{}add a bond from second site in cell [0,0,0] to first site in cell [0,1,0]}
    \PYG{n}{cell}\PYG{o}{.}\PYG{n}{add\PYGZus{}bond}\PYG{p}{(}\PYG{n}{LatticeBond}\PYG{p}{(}\PYG{p}{[}\PYG{l+m+mi}{0}\PYG{p}{,}\PYG{l+m+mi}{0}\PYG{p}{,}\PYG{l+m+mi}{0}\PYG{p}{]}\PYG{p}{,}\PYG{l+m+mi}{1}\PYG{p}{,}\PYG{p}{[}\PYG{l+m+mi}{0}\PYG{p}{,}\PYG{l+m+mi}{1}\PYG{p}{,}\PYG{l+m+mi}{0}\PYG{p}{]}\PYG{p}{,}\PYG{l+m+mi}{0}\PYG{p}{)}\PYG{p}{)}
    \PYG{c+c1}{\PYGZsh{}add a bond from first site in cell [0,0,0] to second site in cell [1,\PYGZhy{}1,0]}
    \PYG{n}{cell}\PYG{o}{.}\PYG{n}{add\PYGZus{}bond}\PYG{p}{(}\PYG{n}{LatticeBond}\PYG{p}{(}\PYG{p}{[}\PYG{l+m+mi}{0}\PYG{p}{,}\PYG{l+m+mi}{0}\PYG{p}{,}\PYG{l+m+mi}{0}\PYG{p}{]}\PYG{p}{,}\PYG{l+m+mi}{0}\PYG{p}{,}\PYG{p}{[}\PYG{l+m+mi}{1}\PYG{p}{,}\PYG{o}{\PYGZhy{}}\PYG{l+m+mi}{1}\PYG{p}{,}\PYG{l+m+mi}{0}\PYG{p}{]}\PYG{p}{,}\PYG{l+m+mi}{1}\PYG{p}{)}\PYG{p}{)}

    \PYG{c+c1}{\PYGZsh{}buld a lattice of shape [4,4,1] with cell we defined}
    \PYG{n}{lattice} \PYG{o}{=} \PYG{n}{Lattice}\PYG{p}{(}\PYG{n}{cell}\PYG{p}{,}\PYG{p}{[}\PYG{l+m+mi}{4}\PYG{p}{,}\PYG{l+m+mi}{4}\PYG{p}{,}\PYG{l+m+mi}{1}\PYG{p}{]}\PYG{p}{)}
    \PYG{k}{return} \PYG{n}{lattice}

\PYG{n}{lattice} \PYG{o}{=} \PYG{n}{graphene}\PYG{p}{(}\PYG{n}{a}\PYG{p}{,}\PYG{n}{a\PYGZus{}cc}\PYG{p}{)}
\PYG{c+c1}{\PYGZsh{}plot the lattice we constructed}
\PYG{n}{lattice}\PYG{o}{.}\PYG{n}{plot}\PYG{p}{(}\PYG{p}{)}
\end{sphinxVerbatim}

Visualize the lattice by \sphinxcode{\sphinxupquote{Lattice.plot()}} method.

\begin{figure}[htbp]
\centering

\noindent\sphinxincludegraphics[scale=1.0]{{graphene}.png}
\end{figure}


\subsubsection{Cubic lattice}
\label{\detokenize{user_model_system:cubic-lattice}}
Final example is a 3d cubic lattice.

\def\sphinxLiteralBlockLabel{\label{\detokenize{user_model_system:id5}}}
\sphinxSetupCaptionForVerbatim{/data/examples/modelsystem/cubic.py}
\fvset{hllines={, ,}}%
\begin{sphinxVerbatim}[commandchars=\\\{\}]
\PYG{k+kn}{from} \PYG{n+nn}{moha} \PYG{k}{import} \PYG{o}{*}

\PYG{n}{d} \PYG{o}{=} \PYG{l+m+mf}{1.0}     \PYG{c+c1}{\PYGZsh{} unit cell length}

\PYG{k}{def} \PYG{n+nf}{cube}\PYG{p}{(}\PYG{n}{d}\PYG{p}{)}\PYG{p}{:}
    \PYG{c+c1}{\PYGZsh{}define a 3D cubic lattice cell with vectors a1 a2 and a3}
    \PYG{n}{cell} \PYG{o}{=} \PYG{n}{Cell}\PYG{p}{(}\PYG{l+m+mi}{3}\PYG{p}{,}\PYG{p}{[}\PYG{n}{d}\PYG{p}{,} \PYG{l+m+mf}{0.}\PYG{p}{,} \PYG{l+m+mf}{0.}\PYG{p}{]}\PYG{p}{,}\PYG{p}{[}\PYG{l+m+mf}{0.}\PYG{p}{,} \PYG{n}{d}\PYG{p}{,} \PYG{l+m+mf}{0.}\PYG{p}{]}\PYG{p}{,}\PYG{p}{[}\PYG{l+m+mf}{0.}\PYG{p}{,} \PYG{l+m+mf}{0.}\PYG{p}{,} \PYG{n}{d}\PYG{p}{]}\PYG{p}{)}

    \PYG{c+c1}{\PYGZsh{}add a site labeled \PYGZsq{}A\PYGZsq{} at positon [0.,0.,0.]}
    \PYG{n}{cell}\PYG{o}{.}\PYG{n}{add\PYGZus{}site}\PYG{p}{(}\PYG{n}{LatticeSite}\PYG{p}{(}\PYG{p}{[}\PYG{l+m+mf}{0.}\PYG{p}{,}\PYG{l+m+mf}{0.}\PYG{p}{,}\PYG{l+m+mf}{0.}\PYG{p}{]}\PYG{p}{,}\PYG{l+s+s1}{\PYGZsq{}}\PYG{l+s+s1}{A}\PYG{l+s+s1}{\PYGZsq{}}\PYG{p}{)}\PYG{p}{)}

    \PYG{c+c1}{\PYGZsh{}add a bond from first site in cell [0,0,0] to second site in cell [0,0,0]}
    \PYG{n}{cell}\PYG{o}{.}\PYG{n}{add\PYGZus{}bond}\PYG{p}{(}\PYG{n}{LatticeBond}\PYG{p}{(}\PYG{p}{[}\PYG{l+m+mi}{0}\PYG{p}{,}\PYG{l+m+mi}{0}\PYG{p}{,}\PYG{l+m+mi}{0}\PYG{p}{]}\PYG{p}{,}\PYG{l+m+mi}{0}\PYG{p}{,}\PYG{p}{[}\PYG{l+m+mi}{1}\PYG{p}{,}\PYG{l+m+mi}{0}\PYG{p}{,}\PYG{l+m+mi}{0}\PYG{p}{]}\PYG{p}{,}\PYG{l+m+mi}{0}\PYG{p}{)}\PYG{p}{)}
    \PYG{c+c1}{\PYGZsh{}add a bond from second site in cell [0,0,0] to first site in cell [0,1,0]}
    \PYG{n}{cell}\PYG{o}{.}\PYG{n}{add\PYGZus{}bond}\PYG{p}{(}\PYG{n}{LatticeBond}\PYG{p}{(}\PYG{p}{[}\PYG{l+m+mi}{0}\PYG{p}{,}\PYG{l+m+mi}{0}\PYG{p}{,}\PYG{l+m+mi}{0}\PYG{p}{]}\PYG{p}{,}\PYG{l+m+mi}{0}\PYG{p}{,}\PYG{p}{[}\PYG{l+m+mi}{0}\PYG{p}{,}\PYG{l+m+mi}{1}\PYG{p}{,}\PYG{l+m+mi}{0}\PYG{p}{]}\PYG{p}{,}\PYG{l+m+mi}{0}\PYG{p}{)}\PYG{p}{)}
    \PYG{c+c1}{\PYGZsh{}add a bond from second site in cell [0,0,0] to first site in cell [0,1,0]}
    \PYG{n}{cell}\PYG{o}{.}\PYG{n}{add\PYGZus{}bond}\PYG{p}{(}\PYG{n}{LatticeBond}\PYG{p}{(}\PYG{p}{[}\PYG{l+m+mi}{0}\PYG{p}{,}\PYG{l+m+mi}{0}\PYG{p}{,}\PYG{l+m+mi}{0}\PYG{p}{]}\PYG{p}{,}\PYG{l+m+mi}{0}\PYG{p}{,}\PYG{p}{[}\PYG{l+m+mi}{0}\PYG{p}{,}\PYG{l+m+mi}{0}\PYG{p}{,}\PYG{l+m+mi}{1}\PYG{p}{]}\PYG{p}{,}\PYG{l+m+mi}{0}\PYG{p}{)}\PYG{p}{)}

    \PYG{c+c1}{\PYGZsh{}buld a lattice of shape [4,4,4] with cell we defined}
    \PYG{n}{lattice} \PYG{o}{=} \PYG{n}{Lattice}\PYG{p}{(}\PYG{n}{cell}\PYG{p}{,}\PYG{p}{[}\PYG{l+m+mi}{4}\PYG{p}{,}\PYG{l+m+mi}{4}\PYG{p}{,}\PYG{l+m+mi}{4}\PYG{p}{]}\PYG{p}{)}
    \PYG{k}{return} \PYG{n}{lattice}

\PYG{n}{lattice} \PYG{o}{=} \PYG{n}{cube}\PYG{p}{(}\PYG{n}{d}\PYG{p}{)}
\PYG{c+c1}{\PYGZsh{}plot the lattice we constructed}
\PYG{n}{lattice}\PYG{o}{.}\PYG{n}{plot}\PYG{p}{(}\PYG{p}{)}
\end{sphinxVerbatim}

Visualize the lattice by \sphinxcode{\sphinxupquote{Lattice.plot()}} method.

\begin{figure}[htbp]
\centering

\noindent\sphinxincludegraphics[scale=1.0]{{cubic}.png}
\end{figure}


\subsection{Model Hamiltonian}
\label{\detokenize{user_model_system:model-hamiltonian}}
Analogy to the construction of molecular Hamiltonian, model hamiltonian usually set up
in three steps.
\begin{itemize}
\item {} 
First, construct a lattice.

\item {} 
Second, generate a basis set for the lattice.To simplifies the mathematics, In what
follows we will assume that the basis set is orthonormal.

\item {} 
Finally, compute all kinds of one body terms and two body terms with that basis
to define a Hamiltonian.

\end{itemize}

With the definition of lattice and site object done, to construct a model Hamiltonian, we need:
\begin{itemize}
\item {} 
Initialize the model Hamiltonian by assign lattice and site

\end{itemize}

\fvset{hllines={, ,}}%
\begin{sphinxVerbatim}[commandchars=\\\{\}]
\PYG{n}{moha} \PYG{o}{=} \PYG{n}{ModelHamiltonian}\PYG{p}{(}\PYG{n}{lattice}\PYG{p}{,}\PYG{n}{site}\PYG{p}{)}
\end{sphinxVerbatim}
\begin{itemize}
\item {} 
Then add one and two body interaction terms

\end{itemize}

\fvset{hllines={, ,}}%
\begin{sphinxVerbatim}[commandchars=\\\{\}]
\PYG{n}{moha}\PYG{o}{.}\PYG{n}{add\PYGZus{}operator}\PYG{p}{(}\PYG{n}{OneBodyTerm}\PYG{p}{(}\PYG{p}{[}\PYG{l+s+s1}{\PYGZsq{}}\PYG{l+s+s1}{c\PYGZus{}dag}\PYG{l+s+s1}{\PYGZsq{}}\PYG{p}{,}\PYG{l+s+s1}{\PYGZsq{}}\PYG{l+s+s1}{c}\PYG{l+s+s1}{\PYGZsq{}}\PYG{p}{]}\PYG{p}{,}\PYG{p}{[}\PYG{l+m+mi}{0}\PYG{p}{,}\PYG{l+m+mi}{0}\PYG{p}{,}\PYG{l+m+mi}{0}\PYG{p}{]}\PYG{p}{,}\PYG{l+m+mi}{0}\PYG{p}{,}\PYG{p}{[}\PYG{l+m+mi}{0}\PYG{p}{,}\PYG{l+m+mi}{0}\PYG{p}{,}\PYG{l+m+mi}{0}\PYG{p}{]}\PYG{p}{,}\PYG{l+m+mi}{0}\PYG{p}{,}\PYG{o}{\PYGZhy{}}\PYG{l+m+mf}{1.0}\PYG{p}{)}\PYG{p}{)}

\PYG{n}{moha}\PYG{o}{.}\PYG{n}{add\PYGZus{}operator}\PYG{p}{(}\PYG{n}{TwoBodyTerm}\PYG{p}{(}\PYG{p}{[}\PYG{l+s+s1}{\PYGZsq{}}\PYG{l+s+s1}{n}\PYG{l+s+s1}{\PYGZsq{}}\PYG{p}{,}\PYG{l+s+s1}{\PYGZsq{}}\PYG{l+s+s1}{n}\PYG{l+s+s1}{\PYGZsq{}}\PYG{p}{]}\PYG{p}{,}\PYG{p}{[}\PYG{l+m+mi}{0}\PYG{p}{,}\PYG{l+m+mi}{0}\PYG{p}{,}\PYG{l+m+mi}{0}\PYG{p}{]}\PYG{p}{,}\PYG{l+m+mi}{0}\PYG{p}{,}\PYG{p}{[}\PYG{l+m+mi}{0}\PYG{p}{,}\PYG{l+m+mi}{0}\PYG{p}{,}\PYG{l+m+mi}{0}\PYG{p}{]}\PYG{p}{,}\PYG{l+m+mi}{0}\PYG{p}{,}\PYG{l+m+mf}{2.0}\PYG{p}{)}\PYG{p}{)}
\end{sphinxVerbatim}

You can check the one body interaction matrix and two body interaction tensor by

\fvset{hllines={, ,}}%
\begin{sphinxVerbatim}[commandchars=\\\{\}]
\PYG{n}{moha}\PYG{o}{.}\PYG{n}{one\PYGZus{}body\PYGZus{}matrix}

\PYG{n}{moha}\PYG{o}{.}\PYG{n}{two\PYGZus{}body\PYGZus{}tensor}
\end{sphinxVerbatim}


\subsubsection{Fermion Model}
\label{\detokenize{user_model_system:fermion-model}}\begin{itemize}
\item {} 
Tight Binding Model

\end{itemize}

In solid-state physics, the tight-binding model (or TB model) is an approach to the calculation
of electronic band structure using an approximate set of wave functions based upon superposition
of wave functions for isolated atoms located at each atomic site.

The Hamiltonian is:
\begin{equation*}
\begin{split}\hat{H} =-t\sum_{<i,j> \sigma}\hat{c}^{\dagger}_{i, \sigma}\hat{c}_{j, \sigma}
   -\mu\sum_{i\sigma}\hat{c}^{\dagger}_{i\sigma}\hat{c}_{i\sigma}\end{split}
\end{equation*}
Often one considers models with only nearest neighbour terms, \(<ij>\) indicates nearest
neighbours. And one takes \(t_{ij}=t\) if i and j are at nearest neighbour sites, and
\(t_{ij}=0\) otherwise, \(\mu\) is the chemical potential.

To represent the Hamilontion in the matrix form, we ignore the spin of the Hamiltonian and transform
the Hamiltonion to
\begin{equation*}
\begin{split}\hat{H} =-t\sum_{<i,j>}\hat{c}^{\dagger}_{i}\hat{c}_{j}
   -\mu\sum_{i}\hat{c}^{\dagger}_{i}\hat{c}_{i}\end{split}
\end{equation*}
\def\sphinxLiteralBlockLabel{\label{\detokenize{user_model_system:id6}}}
\sphinxSetupCaptionForVerbatim{/data/examples/modelsystem/tight\_binding.py}
\fvset{hllines={, ,}}%
\begin{sphinxVerbatim}[commandchars=\\\{\}]
\PYG{k+kn}{from} \PYG{n+nn}{moha} \PYG{k}{import} \PYG{o}{*}
\PYG{k+kn}{from} \PYG{n+nn}{math} \PYG{k}{import} \PYG{n}{sqrt}

\PYG{n}{a} \PYG{o}{=} \PYG{l+m+mf}{0.24595}     \PYG{c+c1}{\PYGZsh{} unit cell length}
\PYG{n}{a\PYGZus{}cc} \PYG{o}{=} \PYG{l+m+mf}{0.142}    \PYG{c+c1}{\PYGZsh{} carbon carbon distance}

\PYG{k}{def} \PYG{n+nf}{graphene}\PYG{p}{(}\PYG{n}{a}\PYG{p}{,}\PYG{n}{a\PYGZus{}cc}\PYG{p}{)}\PYG{p}{:}


    \PYG{c+c1}{\PYGZsh{}define a 2D graphene lattice cell with vectors a1 a2 and a3}
    \PYG{n}{cell} \PYG{o}{=} \PYG{n}{Cell}\PYG{p}{(}\PYG{l+m+mi}{2}\PYG{p}{,}\PYG{p}{[}\PYG{p}{[}\PYG{n}{a}\PYG{p}{,}\PYG{l+m+mf}{0.}\PYG{p}{,}\PYG{l+m+mf}{0.}\PYG{p}{]}\PYG{p}{,}\PYG{p}{[}\PYG{n}{a}\PYG{o}{/}\PYG{l+m+mi}{2}\PYG{p}{,}\PYG{n}{a}\PYG{o}{/}\PYG{l+m+mi}{2}\PYG{o}{*}\PYG{n}{sqrt}\PYG{p}{(}\PYG{l+m+mi}{3}\PYG{p}{)}\PYG{p}{,}\PYG{l+m+mf}{0.}\PYG{p}{]}\PYG{p}{,}\PYG{p}{[}\PYG{l+m+mf}{0.}\PYG{p}{,}\PYG{l+m+mf}{0.}\PYG{p}{,}\PYG{l+m+mf}{0.}\PYG{p}{]}\PYG{p}{]}\PYG{p}{)}


    \PYG{c+c1}{\PYGZsh{}add a site labeled \PYGZsq{}A\PYGZsq{} at positon [0.,\PYGZhy{}a\PYGZus{}cc/2.,0.]}
    \PYG{n}{cell}\PYG{o}{.}\PYG{n}{add\PYGZus{}site}\PYG{p}{(}\PYG{n}{LatticeSite}\PYG{p}{(}\PYG{p}{[}\PYG{l+m+mf}{0.}\PYG{p}{,} \PYG{o}{\PYGZhy{}}\PYG{n}{a\PYGZus{}cc}\PYG{o}{/}\PYG{l+m+mf}{2.}\PYG{p}{,} \PYG{l+m+mf}{0.}\PYG{p}{]}\PYG{p}{,}\PYG{l+s+s1}{\PYGZsq{}}\PYG{l+s+s1}{A}\PYG{l+s+s1}{\PYGZsq{}}\PYG{p}{)}\PYG{p}{)}
    \PYG{c+c1}{\PYGZsh{}add a site labeled \PYGZsq{}B\PYGZsq{} at positon [0.,\PYGZhy{}a\PYGZus{}cc/2.,0.]}
    \PYG{n}{cell}\PYG{o}{.}\PYG{n}{add\PYGZus{}site}\PYG{p}{(}\PYG{n}{LatticeSite}\PYG{p}{(}\PYG{p}{[}\PYG{l+m+mf}{0.}\PYG{p}{,} \PYG{n}{a\PYGZus{}cc}\PYG{o}{/}\PYG{l+m+mf}{2.}\PYG{p}{,} \PYG{l+m+mf}{0.}\PYG{p}{]}\PYG{p}{,}\PYG{l+s+s1}{\PYGZsq{}}\PYG{l+s+s1}{B}\PYG{l+s+s1}{\PYGZsq{}}\PYG{p}{)}\PYG{p}{)}

    \PYG{c+c1}{\PYGZsh{}add a bond from first site in cell [0,0,0] to second site in cell [0,0,0]}
    \PYG{n}{cell}\PYG{o}{.}\PYG{n}{add\PYGZus{}bond}\PYG{p}{(}\PYG{n}{LatticeBond}\PYG{p}{(}\PYG{p}{[}\PYG{l+m+mi}{0}\PYG{p}{,}\PYG{l+m+mi}{0}\PYG{p}{,}\PYG{l+m+mi}{0}\PYG{p}{]}\PYG{p}{,}\PYG{l+m+mi}{0}\PYG{p}{,}\PYG{p}{[}\PYG{l+m+mi}{0}\PYG{p}{,}\PYG{l+m+mi}{0}\PYG{p}{,}\PYG{l+m+mi}{0}\PYG{p}{]}\PYG{p}{,}\PYG{l+m+mi}{1}\PYG{p}{)}\PYG{p}{)}
    \PYG{c+c1}{\PYGZsh{}add a bond from second site in cell [0,0,0] to first site in cell [0,1,0]}
    \PYG{n}{cell}\PYG{o}{.}\PYG{n}{add\PYGZus{}bond}\PYG{p}{(}\PYG{n}{LatticeBond}\PYG{p}{(}\PYG{p}{[}\PYG{l+m+mi}{0}\PYG{p}{,}\PYG{l+m+mi}{0}\PYG{p}{,}\PYG{l+m+mi}{0}\PYG{p}{]}\PYG{p}{,}\PYG{l+m+mi}{1}\PYG{p}{,}\PYG{p}{[}\PYG{l+m+mi}{0}\PYG{p}{,}\PYG{l+m+mi}{1}\PYG{p}{,}\PYG{l+m+mi}{0}\PYG{p}{]}\PYG{p}{,}\PYG{l+m+mi}{0}\PYG{p}{)}\PYG{p}{)}
    \PYG{c+c1}{\PYGZsh{}add a bond from first site in cell [0,0,0] to second site in cell [1,\PYGZhy{}1,0]}
    \PYG{n}{cell}\PYG{o}{.}\PYG{n}{add\PYGZus{}bond}\PYG{p}{(}\PYG{n}{LatticeBond}\PYG{p}{(}\PYG{p}{[}\PYG{l+m+mi}{0}\PYG{p}{,}\PYG{l+m+mi}{0}\PYG{p}{,}\PYG{l+m+mi}{0}\PYG{p}{]}\PYG{p}{,}\PYG{l+m+mi}{0}\PYG{p}{,}\PYG{p}{[}\PYG{l+m+mi}{1}\PYG{p}{,}\PYG{o}{\PYGZhy{}}\PYG{l+m+mi}{1}\PYG{p}{,}\PYG{l+m+mi}{0}\PYG{p}{]}\PYG{p}{,}\PYG{l+m+mi}{1}\PYG{p}{)}\PYG{p}{)}

    \PYG{c+c1}{\PYGZsh{}buld a lattice of shape [4,4,1] with cell we defined}
    \PYG{n}{lattice} \PYG{o}{=} \PYG{n}{Lattice}\PYG{p}{(}\PYG{n}{cell}\PYG{p}{,}\PYG{p}{[}\PYG{l+m+mi}{4}\PYG{p}{,}\PYG{l+m+mi}{4}\PYG{p}{,}\PYG{l+m+mi}{1}\PYG{p}{]}\PYG{p}{)}
    \PYG{k}{return} \PYG{n}{lattice}

\PYG{n}{lattice} \PYG{o}{=} \PYG{n}{graphene}\PYG{p}{(}\PYG{n}{a}\PYG{p}{,}\PYG{n}{a\PYGZus{}cc}\PYG{p}{)}
\PYG{c+c1}{\PYGZsh{}use spinless fermion site as basis}
\PYG{n}{site} \PYG{o}{=} \PYG{n}{SpinlessFermionSite}\PYG{p}{(}\PYG{p}{)}
\PYG{c+c1}{\PYGZsh{}build model hamiltonian}
\PYG{n}{mh} \PYG{o}{=} \PYG{n}{ModelHamiltonian}\PYG{p}{(}\PYG{n}{lattice}\PYG{p}{,}\PYG{n}{site}\PYG{p}{)}
\PYG{c+c1}{\PYGZsh{}add on\PYGZhy{}site term of site A}
\PYG{n}{mh}\PYG{o}{.}\PYG{n}{add\PYGZus{}operator}\PYG{p}{(}\PYG{n}{OneBodyTerm}\PYG{p}{(}\PYG{p}{[}\PYG{l+s+s1}{\PYGZsq{}}\PYG{l+s+s1}{c\PYGZus{}dag}\PYG{l+s+s1}{\PYGZsq{}}\PYG{p}{,}\PYG{l+s+s1}{\PYGZsq{}}\PYG{l+s+s1}{c}\PYG{l+s+s1}{\PYGZsq{}}\PYG{p}{]}\PYG{p}{,}\PYG{p}{[}\PYG{l+m+mi}{0}\PYG{p}{,}\PYG{l+m+mi}{0}\PYG{p}{,}\PYG{l+m+mi}{0}\PYG{p}{]}\PYG{p}{,}\PYG{l+m+mi}{0}\PYG{p}{,}\PYG{p}{[}\PYG{l+m+mi}{0}\PYG{p}{,}\PYG{l+m+mi}{0}\PYG{p}{,}\PYG{l+m+mi}{0}\PYG{p}{]}\PYG{p}{,}\PYG{l+m+mi}{0}\PYG{p}{,}\PYG{o}{\PYGZhy{}}\PYG{l+m+mf}{1.0}\PYG{p}{)}\PYG{p}{)}
\PYG{c+c1}{\PYGZsh{}add on\PYGZhy{}site term of site B}
\PYG{n}{mh}\PYG{o}{.}\PYG{n}{add\PYGZus{}operator}\PYG{p}{(}\PYG{n}{OneBodyTerm}\PYG{p}{(}\PYG{p}{[}\PYG{l+s+s1}{\PYGZsq{}}\PYG{l+s+s1}{c\PYGZus{}dag}\PYG{l+s+s1}{\PYGZsq{}}\PYG{p}{,}\PYG{l+s+s1}{\PYGZsq{}}\PYG{l+s+s1}{c}\PYG{l+s+s1}{\PYGZsq{}}\PYG{p}{]}\PYG{p}{,}\PYG{p}{[}\PYG{l+m+mi}{0}\PYG{p}{,}\PYG{l+m+mi}{0}\PYG{p}{,}\PYG{l+m+mi}{0}\PYG{p}{]}\PYG{p}{,}\PYG{l+m+mi}{1}\PYG{p}{,}\PYG{p}{[}\PYG{l+m+mi}{0}\PYG{p}{,}\PYG{l+m+mi}{0}\PYG{p}{,}\PYG{l+m+mi}{0}\PYG{p}{]}\PYG{p}{,}\PYG{l+m+mi}{1}\PYG{p}{,}\PYG{o}{\PYGZhy{}}\PYG{l+m+mf}{2.0}\PYG{p}{)}\PYG{p}{)}
\PYG{c+c1}{\PYGZsh{}add hopping term of site between site A and B}
\PYG{n}{mh}\PYG{o}{.}\PYG{n}{add\PYGZus{}operator}\PYG{p}{(}\PYG{n}{OneBodyTerm}\PYG{p}{(}\PYG{p}{[}\PYG{l+s+s1}{\PYGZsq{}}\PYG{l+s+s1}{c\PYGZus{}dag}\PYG{l+s+s1}{\PYGZsq{}}\PYG{p}{,}\PYG{l+s+s1}{\PYGZsq{}}\PYG{l+s+s1}{c}\PYG{l+s+s1}{\PYGZsq{}}\PYG{p}{]}\PYG{p}{,}\PYG{p}{[}\PYG{l+m+mi}{0}\PYG{p}{,}\PYG{l+m+mi}{0}\PYG{p}{,}\PYG{l+m+mi}{0}\PYG{p}{]}\PYG{p}{,}\PYG{l+m+mi}{0}\PYG{p}{,}\PYG{p}{[}\PYG{l+m+mi}{0}\PYG{p}{,}\PYG{l+m+mi}{0}\PYG{p}{,}\PYG{l+m+mi}{0}\PYG{p}{]}\PYG{p}{,}\PYG{l+m+mi}{1}\PYG{p}{,}\PYG{o}{\PYGZhy{}}\PYG{l+m+mf}{4.0}\PYG{p}{)}\PYG{p}{)}
\PYG{n}{mh}\PYG{o}{.}\PYG{n}{add\PYGZus{}operator}\PYG{p}{(}\PYG{n}{OneBodyTerm}\PYG{p}{(}\PYG{p}{[}\PYG{l+s+s1}{\PYGZsq{}}\PYG{l+s+s1}{c\PYGZus{}dag}\PYG{l+s+s1}{\PYGZsq{}}\PYG{p}{,}\PYG{l+s+s1}{\PYGZsq{}}\PYG{l+s+s1}{c}\PYG{l+s+s1}{\PYGZsq{}}\PYG{p}{]}\PYG{p}{,}\PYG{p}{[}\PYG{l+m+mi}{0}\PYG{p}{,}\PYG{l+m+mi}{0}\PYG{p}{,}\PYG{l+m+mi}{0}\PYG{p}{]}\PYG{p}{,}\PYG{l+m+mi}{1}\PYG{p}{,}\PYG{p}{[}\PYG{l+m+mi}{0}\PYG{p}{,}\PYG{l+m+mi}{1}\PYG{p}{,}\PYG{l+m+mi}{0}\PYG{p}{]}\PYG{p}{,}\PYG{l+m+mi}{0}\PYG{p}{,}\PYG{o}{\PYGZhy{}}\PYG{l+m+mf}{4.0}\PYG{p}{)}\PYG{p}{)}
\PYG{n}{mh}\PYG{o}{.}\PYG{n}{add\PYGZus{}operator}\PYG{p}{(}\PYG{n}{OneBodyTerm}\PYG{p}{(}\PYG{p}{[}\PYG{l+s+s1}{\PYGZsq{}}\PYG{l+s+s1}{c\PYGZus{}dag}\PYG{l+s+s1}{\PYGZsq{}}\PYG{p}{,}\PYG{l+s+s1}{\PYGZsq{}}\PYG{l+s+s1}{c}\PYG{l+s+s1}{\PYGZsq{}}\PYG{p}{]}\PYG{p}{,}\PYG{p}{[}\PYG{l+m+mi}{0}\PYG{p}{,}\PYG{l+m+mi}{0}\PYG{p}{,}\PYG{l+m+mi}{0}\PYG{p}{]}\PYG{p}{,}\PYG{l+m+mi}{0}\PYG{p}{,}\PYG{p}{[}\PYG{l+m+mi}{1}\PYG{p}{,}\PYG{o}{\PYGZhy{}}\PYG{l+m+mi}{1}\PYG{p}{,}\PYG{l+m+mi}{0}\PYG{p}{]}\PYG{p}{,}\PYG{l+m+mi}{1}\PYG{p}{,}\PYG{o}{\PYGZhy{}}\PYG{l+m+mf}{4.0}\PYG{p}{)}\PYG{p}{)}
\end{sphinxVerbatim}
\begin{itemize}
\item {} 
Hubbard Model

\end{itemize}

The Hubbard model is an approximate model used, especially in solid-state physics, to describle
the transition between conducting and insulating systems. The basic hubbard model have only two
Hamiltonian is the simplest model of interacting particles on a lattice and reads.

The Hamiltonian is:
\begin{equation*}
\begin{split}\hat{H}=-t\sum_{<i,j> \sigma}\hat{c}^{\dagger}_{i, \sigma}\hat{c}_{j, \sigma} +
U\sum_{i}\hat{n}_{i, \uparrow}\hat{n}_{i, \downarrow}\end{split}
\end{equation*}
where the first term is a one-electron term and accounts for the nearest-neighbor hopping, while
the second term is the repulsive on-site interaction. The t and U are user specified parameters
and \(\sigma\) is the electron spin.

To represent the Hamilontion in the matrix form, we ignore the spin of the Hamiltonian and transform
the Hamiltonion to
\begin{equation*}
\begin{split}\hat{H}=-t\sum_{<i,j>}\hat{c}^{\dagger}_{i}\hat{c}_{j} +
U\sum_{i}\hat{n}_{i}\hat{n}_{i}\end{split}
\end{equation*}
\def\sphinxLiteralBlockLabel{\label{\detokenize{user_model_system:id7}}}
\sphinxSetupCaptionForVerbatim{/data/examples/modelsystem/hubbard.py}
\fvset{hllines={, ,}}%
\begin{sphinxVerbatim}[commandchars=\\\{\}]
\PYG{k+kn}{from} \PYG{n+nn}{moha} \PYG{k}{import} \PYG{o}{*}
\PYG{k+kn}{from} \PYG{n+nn}{math} \PYG{k}{import} \PYG{n}{sqrt}

\PYG{n}{a} \PYG{o}{=} \PYG{l+m+mf}{1.0} \PYG{c+c1}{\PYGZsh{}unit cell length}

\PYG{k}{def} \PYG{n+nf}{square}\PYG{p}{(}\PYG{n}{a}\PYG{p}{)}\PYG{p}{:}
    \PYG{c+c1}{\PYGZsh{}define a 2D square lattice cell with vectors a1 a2 and a3}
    \PYG{n}{cell} \PYG{o}{=} \PYG{n}{Cell}\PYG{p}{(}\PYG{l+m+mi}{2}\PYG{p}{,}\PYG{p}{[}\PYG{n}{a}\PYG{p}{,} \PYG{l+m+mf}{0.}\PYG{p}{,} \PYG{l+m+mf}{0.}\PYG{p}{]}\PYG{p}{,}\PYG{p}{[}\PYG{l+m+mf}{0.}\PYG{p}{,} \PYG{n}{a}\PYG{p}{,} \PYG{l+m+mf}{0.}\PYG{p}{]}\PYG{p}{,}\PYG{p}{[}\PYG{l+m+mf}{0.}\PYG{p}{,} \PYG{l+m+mf}{0.}\PYG{p}{,} \PYG{l+m+mf}{0.}\PYG{p}{]}\PYG{p}{)}
    \PYG{c+c1}{\PYGZsh{}add a site labeled \PYGZsq{}A\PYGZsq{} at positon [0.,0.,0.]}
    \PYG{n}{cell}\PYG{o}{.}\PYG{n}{add\PYGZus{}site}\PYG{p}{(}\PYG{n}{LatticeSite}\PYG{p}{(}\PYG{p}{[}\PYG{l+m+mf}{0.}\PYG{p}{,}\PYG{l+m+mf}{0.}\PYG{p}{,}\PYG{l+m+mf}{0.}\PYG{p}{]}\PYG{p}{,}\PYG{l+s+s1}{\PYGZsq{}}\PYG{l+s+s1}{A}\PYG{l+s+s1}{\PYGZsq{}}\PYG{p}{)}\PYG{p}{)}
    \PYG{c+c1}{\PYGZsh{}add a bond from first site in cell [0,0,0] to first site in cell [1,0,0]}
    \PYG{n}{cell}\PYG{o}{.}\PYG{n}{add\PYGZus{}bond}\PYG{p}{(}\PYG{n}{LatticeBond}\PYG{p}{(}\PYG{p}{[}\PYG{l+m+mi}{0}\PYG{p}{,}\PYG{l+m+mi}{0}\PYG{p}{,}\PYG{l+m+mi}{0}\PYG{p}{]}\PYG{p}{,}\PYG{l+m+mi}{0}\PYG{p}{,}\PYG{p}{[}\PYG{l+m+mi}{1}\PYG{p}{,}\PYG{l+m+mi}{0}\PYG{p}{,}\PYG{l+m+mi}{0}\PYG{p}{]}\PYG{p}{,}\PYG{l+m+mi}{0}\PYG{p}{)}\PYG{p}{)}
    \PYG{c+c1}{\PYGZsh{}add a bond from first site in cell [0,0,0] to first site in cell [0,1,0]}
    \PYG{n}{cell}\PYG{o}{.}\PYG{n}{add\PYGZus{}bond}\PYG{p}{(}\PYG{n}{LatticeBond}\PYG{p}{(}\PYG{p}{[}\PYG{l+m+mi}{0}\PYG{p}{,}\PYG{l+m+mi}{0}\PYG{p}{,}\PYG{l+m+mi}{0}\PYG{p}{]}\PYG{p}{,}\PYG{l+m+mi}{0}\PYG{p}{,}\PYG{p}{[}\PYG{l+m+mi}{0}\PYG{p}{,}\PYG{l+m+mi}{1}\PYG{p}{,}\PYG{l+m+mi}{0}\PYG{p}{]}\PYG{p}{,}\PYG{l+m+mi}{0}\PYG{p}{)}\PYG{p}{)}
    \PYG{c+c1}{\PYGZsh{}buld a lattice of shape [4,4,1] with cell we defined}
    \PYG{n}{lattice} \PYG{o}{=} \PYG{n}{Lattice}\PYG{p}{(}\PYG{n}{cell}\PYG{p}{,}\PYG{p}{[}\PYG{l+m+mi}{4}\PYG{p}{,}\PYG{l+m+mi}{4}\PYG{p}{,}\PYG{l+m+mi}{1}\PYG{p}{]}\PYG{p}{)}
    \PYG{k}{return} \PYG{n}{lattice}

\PYG{n}{lattice} \PYG{o}{=} \PYG{n}{square}\PYG{p}{(}\PYG{n}{a}\PYG{p}{)}
\PYG{c+c1}{\PYGZsh{}use spinless fermion site as basis}
\PYG{n}{site} \PYG{o}{=} \PYG{n}{SpinlessFermionSite}\PYG{p}{(}\PYG{p}{)}
\PYG{c+c1}{\PYGZsh{}build model hamiltonian}
\PYG{n}{mh} \PYG{o}{=} \PYG{n}{ModelHamiltonian}\PYG{p}{(}\PYG{n}{lattice}\PYG{p}{,}\PYG{n}{site}\PYG{p}{)}
\PYG{c+c1}{\PYGZsh{}add hopping term of site between nearest neighbour A and B}
\PYG{n}{mh}\PYG{o}{.}\PYG{n}{add\PYGZus{}operator}\PYG{p}{(}\PYG{n}{OneBodyTerm}\PYG{p}{(}\PYG{p}{[}\PYG{l+s+s1}{\PYGZsq{}}\PYG{l+s+s1}{c\PYGZus{}dag}\PYG{l+s+s1}{\PYGZsq{}}\PYG{p}{,}\PYG{l+s+s1}{\PYGZsq{}}\PYG{l+s+s1}{c}\PYG{l+s+s1}{\PYGZsq{}}\PYG{p}{]}\PYG{p}{,}\PYG{p}{[}\PYG{l+m+mi}{0}\PYG{p}{,}\PYG{l+m+mi}{0}\PYG{p}{,}\PYG{l+m+mi}{0}\PYG{p}{]}\PYG{p}{,}\PYG{l+m+mi}{0}\PYG{p}{,}\PYG{p}{[}\PYG{l+m+mi}{1}\PYG{p}{,}\PYG{l+m+mi}{0}\PYG{p}{,}\PYG{l+m+mi}{0}\PYG{p}{]}\PYG{p}{,}\PYG{l+m+mi}{0}\PYG{p}{,}\PYG{o}{\PYGZhy{}}\PYG{l+m+mf}{1.0}\PYG{p}{)}\PYG{p}{)}
\PYG{c+c1}{\PYGZsh{}add two body term of site between site A and B}
\PYG{n}{mh}\PYG{o}{.}\PYG{n}{add\PYGZus{}operator}\PYG{p}{(}\PYG{n}{TwoBodyTerm}\PYG{p}{(}\PYG{p}{[}\PYG{l+s+s1}{\PYGZsq{}}\PYG{l+s+s1}{n}\PYG{l+s+s1}{\PYGZsq{}}\PYG{p}{,}\PYG{l+s+s1}{\PYGZsq{}}\PYG{l+s+s1}{n}\PYG{l+s+s1}{\PYGZsq{}}\PYG{p}{]}\PYG{p}{,}\PYG{p}{[}\PYG{l+m+mi}{0}\PYG{p}{,}\PYG{l+m+mi}{0}\PYG{p}{,}\PYG{l+m+mi}{0}\PYG{p}{]}\PYG{p}{,}\PYG{l+m+mi}{0}\PYG{p}{,}\PYG{p}{[}\PYG{l+m+mi}{0}\PYG{p}{,}\PYG{l+m+mi}{0}\PYG{p}{,}\PYG{l+m+mi}{0}\PYG{p}{]}\PYG{p}{,}\PYG{l+m+mi}{0}\PYG{p}{,}\PYG{l+m+mf}{2.0}\PYG{p}{)}\PYG{p}{)}
\end{sphinxVerbatim}


\subsubsection{Spin Model}
\label{\detokenize{user_model_system:spin-model}}
Quantum spins are a complex object to deal with in many-body physics, it’s niether a canonical
fermions or bosons. However, we can perform the Jordan\textendash{}Wigner transformation that maps spin
operators onto fermionic creation and annihilation operators

where
\begin{equation*}
\begin{split}S^{z}_j = c^{\dagger}_{j}c_{j}-\frac{1}{2}\end{split}
\end{equation*}\begin{equation*}
\begin{split}S^{+}_j = c^{\dagger}_{j}e^{i\pi\sum_{l<j}n_{l}}\end{split}
\end{equation*}\begin{equation*}
\begin{split}S^{-}_j = c_{j}e^{-i\pi\sum_{l<j}n_{l}}\end{split}
\end{equation*}\begin{itemize}
\item {} 
Ising Model

\end{itemize}

The Ising model is a mathematical model of ferromagnetism in statistical mechanics. The
model consists of discrete variables that represent magnetic dipole moments of atomic
spins that can be in one of two states (+1 or −1), allowing each spin to interact
with its neighbors.

The Hamiltonian is:
\begin{equation*}
\begin{split}H=\sum_{<i,j> \sigma}-JS^{z}_{i}S^{z}_{j}\end{split}
\end{equation*}
After Jordan\textendash{}Wigner transformation, it becomes:
\begin{equation*}
\begin{split}H = J\sum_{i}n_{i} - J\sum_{i}n_{i+1}n_{i}\end{split}
\end{equation*}
\def\sphinxLiteralBlockLabel{\label{\detokenize{user_model_system:id8}}}
\sphinxSetupCaptionForVerbatim{/data/examples/modelsystem/ising.py}
\fvset{hllines={, ,}}%
\begin{sphinxVerbatim}[commandchars=\\\{\}]
\PYG{k+kn}{from} \PYG{n+nn}{moha} \PYG{k}{import} \PYG{o}{*}

\PYG{n}{d} \PYG{o}{=} \PYG{l+m+mf}{1.0} \PYG{c+c1}{\PYGZsh{} unit cell length}
\PYG{n}{J} \PYG{o}{=} \PYG{l+m+mf}{1.0} \PYG{c+c1}{\PYGZsh{} spin interaction parameter}

\PYG{k}{def} \PYG{n+nf}{linear}\PYG{p}{(}\PYG{n}{d}\PYG{p}{)}\PYG{p}{:}
    \PYG{c+c1}{\PYGZsh{}define a 1D linear lattice cell with vectors a1 a2 and a3}
    \PYG{n}{cell} \PYG{o}{=} \PYG{n}{Cell}\PYG{p}{(}\PYG{l+m+mi}{1}\PYG{p}{,}\PYG{p}{[}\PYG{n}{d}\PYG{p}{,}\PYG{l+m+mf}{0.}\PYG{p}{,}\PYG{l+m+mf}{0.}\PYG{p}{]}\PYG{p}{,}\PYG{p}{[}\PYG{l+m+mf}{0.}\PYG{p}{,}\PYG{l+m+mf}{0.}\PYG{p}{,}\PYG{l+m+mf}{0.}\PYG{p}{]}\PYG{p}{,}\PYG{p}{[}\PYG{l+m+mf}{0.}\PYG{p}{,}\PYG{l+m+mf}{0.}\PYG{p}{,}\PYG{l+m+mf}{0.}\PYG{p}{]}\PYG{p}{)}
    \PYG{c+c1}{\PYGZsh{}add a site labeled \PYGZsq{}A\PYGZsq{} at positon [0.,0.,0.]}
    \PYG{n}{cell}\PYG{o}{.}\PYG{n}{add\PYGZus{}site}\PYG{p}{(}\PYG{n}{LatticeSite}\PYG{p}{(}\PYG{p}{[}\PYG{l+m+mf}{0.}\PYG{p}{,}\PYG{l+m+mf}{0.}\PYG{p}{,}\PYG{l+m+mf}{0.}\PYG{p}{]}\PYG{p}{,}\PYG{l+s+s1}{\PYGZsq{}}\PYG{l+s+s1}{A}\PYG{l+s+s1}{\PYGZsq{}}\PYG{p}{)}\PYG{p}{)}
    \PYG{c+c1}{\PYGZsh{}add a bond from first site in cell [0,0,0] to first site in cell [1,0,0]}
    \PYG{n}{cell}\PYG{o}{.}\PYG{n}{add\PYGZus{}bond}\PYG{p}{(}\PYG{n}{LatticeBond}\PYG{p}{(}\PYG{p}{[}\PYG{l+m+mi}{0}\PYG{p}{,}\PYG{l+m+mi}{0}\PYG{p}{,}\PYG{l+m+mi}{0}\PYG{p}{]}\PYG{p}{,}\PYG{l+m+mi}{0}\PYG{p}{,}\PYG{p}{[}\PYG{l+m+mi}{1}\PYG{p}{,}\PYG{l+m+mi}{0}\PYG{p}{,}\PYG{l+m+mi}{0}\PYG{p}{]}\PYG{p}{,}\PYG{l+m+mi}{0}\PYG{p}{)}\PYG{p}{)}
    \PYG{c+c1}{\PYGZsh{}buld a lattice of shape [4,1,1] with cell we defined}
    \PYG{n}{lattice} \PYG{o}{=} \PYG{n}{Lattice}\PYG{p}{(}\PYG{n}{cell}\PYG{p}{,}\PYG{p}{[}\PYG{l+m+mi}{4}\PYG{p}{,}\PYG{l+m+mi}{1}\PYG{p}{,}\PYG{l+m+mi}{1}\PYG{p}{]}\PYG{p}{)}
    \PYG{k}{return} \PYG{n}{lattice}

\PYG{n}{lattice} \PYG{o}{=} \PYG{n}{linear}\PYG{p}{(}\PYG{n}{d}\PYG{p}{)}
\PYG{n}{site} \PYG{o}{=} \PYG{n}{SpinlessFermionSite}\PYG{p}{(}\PYG{p}{)}
\PYG{n}{moha} \PYG{o}{=} \PYG{n}{ModelHamiltonian}\PYG{p}{(}\PYG{n}{lattice}\PYG{p}{,}\PYG{n}{site}\PYG{p}{)}
\PYG{n}{moha}\PYG{o}{.}\PYG{n}{add\PYGZus{}operator}\PYG{p}{(}\PYG{n}{OneBodyTerm}\PYG{p}{(}\PYG{p}{[}\PYG{l+s+s1}{\PYGZsq{}}\PYG{l+s+s1}{n}\PYG{l+s+s1}{\PYGZsq{}}\PYG{p}{]}\PYG{p}{,}\PYG{p}{[}\PYG{l+m+mi}{0}\PYG{p}{,}\PYG{l+m+mi}{0}\PYG{p}{,}\PYG{l+m+mi}{0}\PYG{p}{]}\PYG{p}{,}\PYG{l+m+mi}{0}\PYG{p}{,}\PYG{p}{[}\PYG{l+m+mi}{0}\PYG{p}{,}\PYG{l+m+mi}{0}\PYG{p}{,}\PYG{l+m+mi}{0}\PYG{p}{]}\PYG{p}{,}\PYG{l+m+mi}{0}\PYG{p}{,}\PYG{n}{J}\PYG{p}{)}\PYG{p}{)}
\PYG{n}{moha}\PYG{o}{.}\PYG{n}{add\PYGZus{}operator}\PYG{p}{(}\PYG{n}{TwoBodyTerm}\PYG{p}{(}\PYG{p}{[}\PYG{l+s+s1}{\PYGZsq{}}\PYG{l+s+s1}{n}\PYG{l+s+s1}{\PYGZsq{}}\PYG{p}{,}\PYG{l+s+s1}{\PYGZsq{}}\PYG{l+s+s1}{n}\PYG{l+s+s1}{\PYGZsq{}}\PYG{p}{]}\PYG{p}{,}\PYG{p}{[}\PYG{l+m+mi}{0}\PYG{p}{,}\PYG{l+m+mi}{0}\PYG{p}{,}\PYG{l+m+mi}{0}\PYG{p}{]}\PYG{p}{,}\PYG{l+m+mi}{0}\PYG{p}{,}\PYG{p}{[}\PYG{l+m+mi}{1}\PYG{p}{,}\PYG{l+m+mi}{0}\PYG{p}{,}\PYG{l+m+mi}{0}\PYG{p}{]}\PYG{p}{,}\PYG{l+m+mi}{0}\PYG{p}{,}\PYG{o}{\PYGZhy{}}\PYG{n}{J}\PYG{p}{)}\PYG{p}{)}
\end{sphinxVerbatim}
\begin{itemize}
\item {} 
Heisenberg Model

\end{itemize}

The Heisenberg model is a statistical mechanical model used in the study of critical points
and phase transitions of magnetic systems, in which the spins of the magnetic systems are
treated quantum mechanically.

The Hamiltonian is:
\begin{equation*}
\begin{split}H=J\sum_{<i,j>}\vec{S}_{i}\vec{S}_{j} =
J\sum_{<ij>}\left[S^{z}_{i}S^{z}_{j} +
\frac{1}{2}\left(S^{+}_{i}S^{-}_{j}+S^{-}_{i}S^{+}_{j}\right)\right]\end{split}
\end{equation*}
where
\begin{equation*}
\begin{split}\hat{S}_{i}^{+} = \hat{S}_{i}^{x} + i\hat{S}_{i}^{y}\end{split}
\end{equation*}\begin{equation*}
\begin{split}\hat{S}_{i}^{-} = \hat{S}_{i}^{x} - i\hat{S}_{i}^{y}\end{split}
\end{equation*}\begin{equation*}
\begin{split}\hat{S}_{i}^{z} = \frac{1}{2}(\hat{S}_{i}^{+}\hat{S}_{i}^{-}-\hat{S}_{i}^{-}\hat{S}_{i}^{+})\end{split}
\end{equation*}
To perform the Jordan-Wigner transformation
\begin{equation*}
\begin{split}H=-\frac{J}{2}\sum_{i}\left(c^{\dagger}_{i+1}c_{i}+c^{\dagger}_{i}c_{i+1}\right)
+J\sum_{i}n_{i} - J\sum_{i}n_{i+1}n_{i}\end{split}
\end{equation*}
\def\sphinxLiteralBlockLabel{\label{\detokenize{user_model_system:id9}}}
\sphinxSetupCaptionForVerbatim{/data/examples/modelsystem/heisenberg.py}
\fvset{hllines={, ,}}%
\begin{sphinxVerbatim}[commandchars=\\\{\}]
\PYG{k+kn}{from} \PYG{n+nn}{moha} \PYG{k}{import} \PYG{o}{*}

\PYG{n}{d} \PYG{o}{=} \PYG{l+m+mf}{1.0} \PYG{c+c1}{\PYGZsh{} unit cell length}
\PYG{n}{J} \PYG{o}{=} \PYG{l+m+mf}{1.0} \PYG{c+c1}{\PYGZsh{} spin interaction parameter}

\PYG{k}{def} \PYG{n+nf}{linear}\PYG{p}{(}\PYG{n}{d}\PYG{p}{)}\PYG{p}{:}
    \PYG{c+c1}{\PYGZsh{}define a 1D linear lattice cell with vectors a1 a2 and a3}
    \PYG{n}{cell} \PYG{o}{=} \PYG{n}{Cell}\PYG{p}{(}\PYG{l+m+mi}{1}\PYG{p}{,}\PYG{p}{[}\PYG{n}{d}\PYG{p}{,}\PYG{l+m+mf}{0.}\PYG{p}{,}\PYG{l+m+mf}{0.}\PYG{p}{]}\PYG{p}{,}\PYG{p}{[}\PYG{l+m+mf}{0.}\PYG{p}{,}\PYG{l+m+mf}{0.}\PYG{p}{,}\PYG{l+m+mf}{0.}\PYG{p}{]}\PYG{p}{,}\PYG{p}{[}\PYG{l+m+mf}{0.}\PYG{p}{,}\PYG{l+m+mf}{0.}\PYG{p}{,}\PYG{l+m+mf}{0.}\PYG{p}{]}\PYG{p}{)}
    \PYG{c+c1}{\PYGZsh{}add a site labeled \PYGZsq{}A\PYGZsq{} at positon [0.,0.,0.]}
    \PYG{n}{cell}\PYG{o}{.}\PYG{n}{add\PYGZus{}site}\PYG{p}{(}\PYG{n}{LatticeSite}\PYG{p}{(}\PYG{p}{[}\PYG{l+m+mf}{0.}\PYG{p}{,}\PYG{l+m+mf}{0.}\PYG{p}{,}\PYG{l+m+mf}{0.}\PYG{p}{]}\PYG{p}{,}\PYG{l+s+s1}{\PYGZsq{}}\PYG{l+s+s1}{A}\PYG{l+s+s1}{\PYGZsq{}}\PYG{p}{)}\PYG{p}{)}
    \PYG{c+c1}{\PYGZsh{}add a bond from first site in cell [0,0,0] to first site in cell [1,0,0]}
    \PYG{n}{cell}\PYG{o}{.}\PYG{n}{add\PYGZus{}bond}\PYG{p}{(}\PYG{n}{LatticeBond}\PYG{p}{(}\PYG{p}{[}\PYG{l+m+mi}{0}\PYG{p}{,}\PYG{l+m+mi}{0}\PYG{p}{,}\PYG{l+m+mi}{0}\PYG{p}{]}\PYG{p}{,}\PYG{l+m+mi}{0}\PYG{p}{,}\PYG{p}{[}\PYG{l+m+mi}{1}\PYG{p}{,}\PYG{l+m+mi}{0}\PYG{p}{,}\PYG{l+m+mi}{0}\PYG{p}{]}\PYG{p}{,}\PYG{l+m+mi}{0}\PYG{p}{)}\PYG{p}{)}
    \PYG{c+c1}{\PYGZsh{}buld a lattice of shape [4,1,1] with cell we defined}
    \PYG{n}{lattice} \PYG{o}{=} \PYG{n}{Lattice}\PYG{p}{(}\PYG{n}{cell}\PYG{p}{,}\PYG{p}{[}\PYG{l+m+mi}{4}\PYG{p}{,}\PYG{l+m+mi}{1}\PYG{p}{,}\PYG{l+m+mi}{1}\PYG{p}{]}\PYG{p}{)}
    \PYG{k}{return} \PYG{n}{lattice}

\PYG{n}{lattice} \PYG{o}{=} \PYG{n}{linear}\PYG{p}{(}\PYG{n}{d}\PYG{p}{)}
\PYG{n}{site} \PYG{o}{=} \PYG{n}{SpinlessFermionSite}\PYG{p}{(}\PYG{p}{)}
\PYG{n}{moha} \PYG{o}{=} \PYG{n}{ModelHamiltonian}\PYG{p}{(}\PYG{n}{lattice}\PYG{p}{,}\PYG{n}{site}\PYG{p}{)}
\PYG{n}{moha}\PYG{o}{.}\PYG{n}{add\PYGZus{}operator}\PYG{p}{(}\PYG{n}{OneBodyTerm}\PYG{p}{(}\PYG{p}{[}\PYG{l+s+s1}{\PYGZsq{}}\PYG{l+s+s1}{c\PYGZus{}dag}\PYG{l+s+s1}{\PYGZsq{}}\PYG{p}{,}\PYG{l+s+s1}{\PYGZsq{}}\PYG{l+s+s1}{c}\PYG{l+s+s1}{\PYGZsq{}}\PYG{p}{]}\PYG{p}{,}\PYG{p}{[}\PYG{l+m+mi}{0}\PYG{p}{,}\PYG{l+m+mi}{0}\PYG{p}{,}\PYG{l+m+mi}{0}\PYG{p}{]}\PYG{p}{,}\PYG{l+m+mi}{0}\PYG{p}{,}\PYG{p}{[}\PYG{l+m+mi}{1}\PYG{p}{,}\PYG{l+m+mi}{0}\PYG{p}{,}\PYG{l+m+mi}{0}\PYG{p}{]}\PYG{p}{,}\PYG{l+m+mi}{0}\PYG{p}{,}\PYG{o}{\PYGZhy{}}\PYG{n}{J}\PYG{o}{/}\PYG{l+m+mf}{2.}\PYG{p}{)}\PYG{p}{)}
\PYG{n}{moha}\PYG{o}{.}\PYG{n}{add\PYGZus{}operator}\PYG{p}{(}\PYG{n}{OneBodyTerm}\PYG{p}{(}\PYG{p}{[}\PYG{l+s+s1}{\PYGZsq{}}\PYG{l+s+s1}{n}\PYG{l+s+s1}{\PYGZsq{}}\PYG{p}{]}\PYG{p}{,}\PYG{p}{[}\PYG{l+m+mi}{0}\PYG{p}{,}\PYG{l+m+mi}{0}\PYG{p}{,}\PYG{l+m+mi}{0}\PYG{p}{]}\PYG{p}{,}\PYG{l+m+mi}{0}\PYG{p}{,}\PYG{p}{[}\PYG{l+m+mi}{0}\PYG{p}{,}\PYG{l+m+mi}{0}\PYG{p}{,}\PYG{l+m+mi}{0}\PYG{p}{]}\PYG{p}{,}\PYG{l+m+mi}{0}\PYG{p}{,}\PYG{n}{J}\PYG{p}{)}\PYG{p}{)}
\PYG{n}{moha}\PYG{o}{.}\PYG{n}{add\PYGZus{}operator}\PYG{p}{(}\PYG{n}{TwoBodyTerm}\PYG{p}{(}\PYG{p}{[}\PYG{l+s+s1}{\PYGZsq{}}\PYG{l+s+s1}{n}\PYG{l+s+s1}{\PYGZsq{}}\PYG{p}{,}\PYG{l+s+s1}{\PYGZsq{}}\PYG{l+s+s1}{n}\PYG{l+s+s1}{\PYGZsq{}}\PYG{p}{]}\PYG{p}{,}\PYG{p}{[}\PYG{l+m+mi}{0}\PYG{p}{,}\PYG{l+m+mi}{0}\PYG{p}{,}\PYG{l+m+mi}{0}\PYG{p}{]}\PYG{p}{,}\PYG{l+m+mi}{0}\PYG{p}{,}\PYG{p}{[}\PYG{l+m+mi}{1}\PYG{p}{,}\PYG{l+m+mi}{0}\PYG{p}{,}\PYG{l+m+mi}{0}\PYG{p}{]}\PYG{p}{,}\PYG{l+m+mi}{0}\PYG{p}{,}\PYG{o}{\PYGZhy{}}\PYG{n}{J}\PYG{p}{)}\PYG{p}{)}
\end{sphinxVerbatim}
\begin{itemize}
\item {} 
XXZ Model

\end{itemize}

The Hamiltonian is:
\begin{equation*}
\begin{split}H=\sum_{<ij>}J_{z}S^{z}_{i}S^{z}_{j} +
\frac{J}{2}\left(S^{+}_{i}S^{-}_{j}+S^{-}_{i}S^{+}_{j}\right)\end{split}
\end{equation*}
To perform the Jordan-Wigner transformation
\begin{equation*}
\begin{split}H=-\frac{J}{2}\sum_{i}\left(c^{\dagger}_{i+1}c_{i}+c^{\dagger}_{i}c_{i+1}\right)
+J_{z}\sum_{i}n_{i} - J_{z}\sum_{i}n_{i+1}n_{i}\end{split}
\end{equation*}\begin{itemize}
\item {} 
XY Model

\end{itemize}

The Hamiltonian is:
\begin{equation*}
\begin{split}H=\sum_{<ij>}\frac{J}{2}\left(S^{+}_{i}S^{-}_{j}+S^{-}_{i}S^{+}_{j}\right)\end{split}
\end{equation*}
To perform the Jordan-Wigner transformation
\begin{equation*}
\begin{split}H=-\frac{J}{2}\sum_{i}\left(c^{\dagger}_{i+1}c_{i}+c^{\dagger}_{i}c_{i+1}\right)\end{split}
\end{equation*}

\subsection{Program reference}
\label{\detokenize{user_model_system:module-moha.modelsystem}}\label{\detokenize{user_model_system:program-reference}}\index{moha.modelsystem (module)}

\subsubsection{Non-relativistic Hartree-Fock}
\label{\detokenize{user_model_system:non-relativistic-hartree-fock}}

\section{SCF}
\label{\detokenize{posthf:scf}}\label{\detokenize{posthf::doc}}
To begin a calculation with MoHa, the first step is to build a Hamiltonian of a
system, either molecular system or model system. In most cases, we need to build a molecular
system

In terms of second quantisation operators, a general Hamiltonian can be written
as
\begin{equation*}
\begin{split}H = - \sum_{ij} t_{ij}\hat{c}^{\dagger}_{i}\hat{c}_{j} + \frac{1}{2} \sum_{ijkl}
V_{ijkl}\hat{c}^{\dagger}_{i}\hat{c}^{\dagger}_{k}\hat{c}_{l}\hat{c}_{j}\end{split}
\end{equation*}
The construction of molecular Hamiltonian usually set up in three steps.
\begin{itemize}
\item {} 
First, construct a molecular geometry.

\item {} 
Second, generate a Gaussian basis set for the molecular.

\item {} 
Finally, compute all kinds of one body terms and two body terms with that basis
to define a Hamiltonian.

\end{itemize}


\section{Properties}
\label{\detokenize{properties:properties}}\label{\detokenize{properties:properties-rst}}\label{\detokenize{properties::doc}}
This section containts information about atomic and molecular properties that can be calculated.


\subsection{Population Analysis/Atomic Charges}
\label{\detokenize{properties:population-analysis-atomic-charges}}

\subsection{Multipole Moment}
\label{\detokenize{properties:multipole-moment}}

\subsection{Random-Phase Approximation Excitation energy}
\label{\detokenize{properties:random-phase-approximation-excitation-energy}}

\section{API Doucmentation}
\label{\detokenize{developer_api:api-doucmentation}}\label{\detokenize{developer_api::doc}}
This part of the documentation is generated from the docstrings in the source
code.


\section{API Doucmentation}
\label{\detokenize{developer_derivation:api-doucmentation}}\label{\detokenize{developer_derivation::doc}}
To begin a calculation with MoHa, the first step is to build a Hamiltonian of a
system, either molecular system or model system. In most cases, we need to build a molecular
system

In terms of second quantisation operators, a general Hamiltonian can be written
as
1
.. math:

\fvset{hllines={, ,}}%
\begin{sphinxVerbatim}[commandchars=\\\{\}]
\PYG{n}{H} \PYG{o}{=} \PYG{o}{\PYGZhy{}} \PYGZbs{}\PYG{n}{sum\PYGZus{}}\PYG{p}{\PYGZob{}}\PYG{n}{ij}\PYG{p}{\PYGZcb{}} \PYG{n}{t\PYGZus{}}\PYG{p}{\PYGZob{}}\PYG{n}{ij}\PYG{p}{\PYGZcb{}}\PYGZbs{}\PYG{n}{hat}\PYG{p}{\PYGZob{}}\PYG{n}{c}\PYG{p}{\PYGZcb{}}\PYG{o}{\PYGZca{}}\PYG{p}{\PYGZob{}}\PYGZbs{}\PYG{n}{dagger}\PYG{p}{\PYGZcb{}}\PYG{n}{\PYGZus{}}\PYG{p}{\PYGZob{}}\PYG{n}{i}\PYG{p}{\PYGZcb{}}\PYGZbs{}\PYG{n}{hat}\PYG{p}{\PYGZob{}}\PYG{n}{c}\PYG{p}{\PYGZcb{}}\PYG{n}{\PYGZus{}}\PYG{p}{\PYGZob{}}\PYG{n}{j}\PYG{p}{\PYGZcb{}} \PYG{o}{+} \PYGZbs{}\PYG{n}{frac}\PYG{p}{\PYGZob{}}\PYG{l+m+mi}{1}\PYG{p}{\PYGZcb{}}\PYG{p}{\PYGZob{}}\PYG{l+m+mi}{2}\PYG{p}{\PYGZcb{}} \PYGZbs{}\PYG{n}{sum\PYGZus{}}\PYG{p}{\PYGZob{}}\PYG{n}{ijkl}\PYG{p}{\PYGZcb{}}
\PYG{n}{V\PYGZus{}}\PYG{p}{\PYGZob{}}\PYG{n}{ijkl}\PYG{p}{\PYGZcb{}}\PYGZbs{}\PYG{n}{hat}\PYG{p}{\PYGZob{}}\PYG{n}{c}\PYG{p}{\PYGZcb{}}\PYG{o}{\PYGZca{}}\PYG{p}{\PYGZob{}}\PYGZbs{}\PYG{n}{dagger}\PYG{p}{\PYGZcb{}}\PYG{n}{\PYGZus{}}\PYG{p}{\PYGZob{}}\PYG{n}{i}\PYG{p}{\PYGZcb{}}\PYGZbs{}\PYG{n}{hat}\PYG{p}{\PYGZob{}}\PYG{n}{c}\PYG{p}{\PYGZcb{}}\PYG{o}{\PYGZca{}}\PYG{p}{\PYGZob{}}\PYGZbs{}\PYG{n}{dagger}\PYG{p}{\PYGZcb{}}\PYG{n}{\PYGZus{}}\PYG{p}{\PYGZob{}}\PYG{n}{k}\PYG{p}{\PYGZcb{}}\PYGZbs{}\PYG{n}{hat}\PYG{p}{\PYGZob{}}\PYG{n}{c}\PYG{p}{\PYGZcb{}}\PYG{n}{\PYGZus{}}\PYG{p}{\PYGZob{}}\PYG{n}{l}\PYG{p}{\PYGZcb{}}\PYGZbs{}\PYG{n}{hat}\PYG{p}{\PYGZob{}}\PYG{n}{c}\PYG{p}{\PYGZcb{}}\PYG{n}{\PYGZus{}}\PYG{p}{\PYGZob{}}\PYG{n}{j}\PYG{p}{\PYGZcb{}}
\end{sphinxVerbatim}

The construction of molecular Hamiltonian usually set up in three steps.
\begin{itemize}
\item {} 
First, construct a molecular geometry.

\item {} 
Second, generate a Gaussian basis set for the molecular.

\item {} 
Finally, compute all kinds of one body terms and two body terms with that basis
to define a Hamiltonian.

\end{itemize}


\subsection{Molecule}
\label{\detokenize{developer_derivation:molecule}}
fdfdsfs


\chapter{Indices and tables}
\label{\detokenize{index:indices-and-tables}}\begin{itemize}
\item {} 
\DUrole{xref,std,std-ref}{genindex}

\item {} 
\DUrole{xref,std,std-ref}{modindex}

\end{itemize}


\renewcommand{\indexname}{Python Module Index}
\begin{sphinxtheindex}
\def\bigletter#1{{\Large\sffamily#1}\nopagebreak\vspace{1mm}}
\bigletter{m}
\item {\sphinxstyleindexentry{moha.modelsystem}}\sphinxstyleindexpageref{user_model_system:\detokenize{module-moha.modelsystem}}
\end{sphinxtheindex}

\renewcommand{\indexname}{Index}
\printindex
\end{document}